\documentclass{article}
\usepackage{ltcadiz}
\begin{document}

This is a specification of a simple scheduler and assembler. The
system contains a set of registers and a block of memory. Processes
can be created, with each containing a sequence on instructions that
are executed on the system. The instruction format is a simplified
format of the Intel x86 architecture. Processes are scheduled based on
the credit system that is found in the Linux 2.0 kernel.

This specification was written as a test spec for the CZT project. As
a result, there are parts that may appear to be specified in a strange
way - this is to test out the tools on a large set of Z.

\begin{zsection}
  \SECTION Stack \parents standard\_toolkit
\end{zsection}

A generic stack.

\begin{zed}
  Stack[X] == [stack : \seq X]\\
  InitStack[X] == [Stack[X] | stack = \emptyset]
\end{zed}

\begin{schema}{PushStack}[X]
  \Delta Stack[X]\\
  x? : X
\where
  stack' = stack \cat \langle x? \rangle
\end{schema}

\begin{schema}{PopStack}[X]
  \Delta Stack[X]\\
  x! : X
\where
  stack' \cat \langle x! \rangle = stack
\end{schema}


\begin{zsection}
  \SECTION Definitions \parents standard\_toolkit
\end{zsection}

Firstly, we define some basic types and functions that are used
throughout the specification.

$singleton$ is the set of all sets whose size is less than or equal to~$1$.
This is included only to have a generic axiom definition.

\begin{zed}
  \relation (singleton~\_~)
\end{zed}

\begin{gendef}[X]
  singleton~\_ : \power (\power X)
\where
  \forall s : \power X @ singleton~s \iff \#s \leq 1
\end{gendef}

The basic type of this system is a word, which specifically, is an
unsigned octet. An unsigned word is used so references to memory etc a
1-relative.

\begin{zed}
  WORD == 0 \upto 255
\end{zed}

Then, we define the size of the memory block, and give it a value for
animation purposes.

\begin{axdef}
  mem\_size : WORD
\where
  mem\_size = 100
\end{axdef}

A $LABEL$ is used to label instructions for $jump$ instructions etc,
although 'jump' hasn't been specified yet.

\begin{zed}
  [LABEL]
\end{zed}

Now we define the different instructions, as well as their operands. A
$CONSTANT$ is used both as a constant value, as well as a memory
reference for load and store instructions.

\begin{zed}
  INST\_NAME ::= \\
    \t1 add | sub | divide | mult | push | pop |
      load | store | loadConst | print \\
  OPERAND ::= AX | BX | CX | DX | constant \ldata WORD \rdata\\
  REGISTER == \{ AX, BX, CX, DX \}\\
  CONSTANT == OPERAND \setminus REGISTER
\end{zed}

An instruction is specified as a instruction name, a sequence of
operands, and optionally, a label.

\begin{schema}{Instruction}
  label : \power LABEL \\
  name : INST\_NAME \\
  params : \seq OPERAND
\where
  singleton~label
\end{schema}

\begin{zsection}
  \SECTION System \parents Definitions, Stack
\end{zsection}

The system consists of a set of registers, and a block of
memory. There is also a buffer for displaying output.

\begin{zed}
  REGISTERS == REGISTER \fun OPERAND\\
  MEMORY == 1 \upto mem\_size \pfun WORD    
\end{zed}

\begin{schema}{System}
  registers : REGISTERS\\
  memory : MEMORY\\
  output : \seq WORD
\end{schema}

Initially, all registers and memory hold the minimum $WORD$ value. The
output buffer is empty.

\begin{schema}{InitSystem}
  System
\where
  registers = \{ r : REGISTER @ r \mapsto constant(min(WORD)) \}\\
  memory = \{ m : 1 \upto mem\_size @ m \mapsto min(WORD) \}\\
  output = \langle \rangle
\end{schema}

The system can have arithmetic and memory instructions.

\begin{zed}
  Arith\_Inst == [Instruction | \# params = 2 \land params(1) \in REGISTER]\\
  Add\_Inst == [Arith\_Inst| name = add]\\
  Sub\_Inst == [Arith\_Inst | name = sub]\\
  Mult\_Inst == [Arith\_Inst | name = mult]\\
  Div\_Inst == [Arith\_Inst | name = divide]
\end{zed}

\begin{zed}
  Memory\_Inst == [Instruction | \# params = 2 \land params(1) \in REGISTER \\
    \t1 \land params(2) \in CONSTANT]\\
  Load\_Inst == [Memory\_Inst | name = load]\\
  LoadConst\_Inst == [Memory\_Inst | name = loadConst]\\
  Store\_Inst == [Memory\_Inst | name = store]
\end{zed}

A print instruction prints the value of a register.

\begin{zed}
  Print\_Inst == [Instruction | \# params = 1]
\end{zed}

$val$ maps constants to their value, and $dereference$ dereferences
the value of a register, transitively if required.

\begin{axdef}
  val : CONSTANT \fun WORD\\
  dereference : OPERAND \cross REGISTERS \fun WORD
\where
  \forall c : CONSTANT @ \\
    \t1 (\exists n : WORD @ c = constant(n) \land val(c) = n)\\
  \forall a : OPERAND ; r : REGISTERS @ \\
    \t1 dereference(a,r) = \\
      \t2 \IF a \in REGISTER \THEN dereference(r(a),r) \ELSE val(a)
\end{axdef}

The specification of the arithmetic instructions.

\begin{schema}{Add}
  \Delta System\\
  Add\_Inst
\where
  \exists o_1 == dereference(params(1), registers);\\
    \t1 o_2 == dereference(params(2), registers) @\\
    \t1 registers' = registers \oplus 
      \{ params(1) \mapsto constant(o_1 + o_2) \}\\
  memory' = memory\\
  output' = output
\end{schema}

\begin{schema}{Sub}
  \Delta System\\
  Sub\_Inst
\where
  \exists o_1 == dereference(params(1), registers);\\
    \t1 o_2 == dereference(params(2), registers) @\\
    \t1 registers' = registers \oplus 
      \{ params(1) \mapsto constant(o_1 - o_2) \}\\
  memory' = memory\\
  output' = output
\end{schema}

\begin{schema}{Mult}
  \Delta System\\
  Mult\_Inst
\where
  \exists o_1 == dereference(params(1), registers);\\
    \t1 o_2 == dereference(params(2), registers) @\\
    \t1 registers' = registers \oplus 
      \{ params(1) \mapsto constant(o_1 * o_2) \}\\
  memory' = memory\\
  output' = output
\end{schema}

\begin{schema}{Div}
  \Delta System\\
  Div\_Inst
\where
  \exists o_1 == dereference(params(1), registers);\\
    \t1 o_2 == dereference(params(2), registers) @\\
    \t1 registers' = registers \oplus 
      \{ params(1) \mapsto constant(o_1 \div o_2) \}\\
  memory' = memory\\
  output' = output
\end{schema}

The {\tt load} operation loads a constant from memory. The second
parameter is an index to the memory location from which the constant
is loaded.

\begin{schema}{Load}
  \Delta System\\
  Load\_Inst
\where
  \exists o_2 == val(params(2)) @\\
    \t1 registers' = registers \oplus \\
      \t2 \{ params(1) \mapsto constant(memory(o_2)) \}\\
  memory' = memory\\
  output' = output
\end{schema}

{\tt loadConst} loads a constant into a register. The second parameter
the constant to be loaded.

\begin{schema}{Load\_Const}
  \Delta System\\
  LoadConst\_Inst
\where
  \exists o_2 == val(params(2)) @\\
    \t1 registers' = registers \oplus \{ params(1) \mapsto constant(o_2) \}\\
  memory' = memory\\
  output' = output
\end{schema}

Store the value of a register in memory.

\begin{schema}{Store}
  \Delta System\\
  Store\_Inst
\where
  \exists o_1 == dereference(params(1), registers);\\
    \t1 o_2 == val(params(2)) @\\
      \t2 memory' = memory \oplus \{ o_2 \mapsto o_1 \}\\
  registers' = registers\\
  output' = output
\end{schema}

\begin{schema}{Print}
  \Xi System\\
  Print\_Inst
\where
  output' = output \cat \langle dereference(params(1), registers) \rangle\\
  registers' = registers\\
  memory' = memory
\end{schema}

\begin{zed}
  Stack\_Inst == [Instruction | \#params = 1 ]\\
  Push\_Inst == [Stack\_Inst | name = push ]\\
  Pop\_Inst == [Stack\_Inst | name = pop]
\end{zed}

The specification of the stack instructions on the system.

\begin{schema}{Push0}
  \Xi System\\
  PushStack[WORD]\\
  Push\_Inst
\where
  x? = dereference(params(1), registers)
\end{schema}

\begin{schema}{Pop0}
  \Delta System\\
  PopStack[WORD]\\
  Pop\_Inst
\where
  registers' = registers \oplus \{ params(1) \mapsto constant(x!) \}\\
  memory' = memory\\
  output' = output
\end{schema}

\begin{zed}
  Push == Push0 \project [ System ; Stack [WORD] ]\\
  Pop == Pop0 \project [System ; Stack [WORD] ]
\end{zed}

This executes an instruction on the on the system. $inst?$ is the
instruction to execute, and $base?$ is the base memory value of the
executing process. If the instruction is a {\tt load} or {\tt store}
instruction, the memory reference must offset using the base value.

\begin{schema}{exec\_inst}
  \Delta System\\
  inst? : Instruction\\
  base? : 1 \upto mem\_size
\where
  \exists label : \power LABEL; name : INST\_NAME; params : \seq OPERAND |\\
    \t1 label = inst?.label \land name = inst?.name \land \\
    \t1 params = inst?.params @\\
      \t2 Add \lor Sub \lor Mult \lor Div \lor\\
      \t2 Print \lor Load\_Const \lor\\
      \t2 name \in \{load,store\} \implies (\exists p : \seq OPERAND |\\
        \t3 p = \langle params(1), \\
          \t4 constant(val(params(2))+base?) \rangle @\\
        \t3 Load [p/params] \lor Store [p/params] )
\end{schema}

\begin{zsection}
  \SECTION Scheduler \parents System
\end{zsection}

This part of the specification is the scheduler.

Here, we declare the set of process IDs, the priority values, and the
default number of credits a process receives when it is created.

\begin{zed}
  Pid == \nat\\
  Priority == \negate 19 \upto 19\\
  Default\_Credits == 10 
\end{zed}

The possible status that a process can hold.

\begin{zed}
  Status ::= pWaiting | pReady | pRunning
\end{zed}

A process consists of a process ID, a status, a number of credits, and
a priority. Each process has a sequence of instructions to be executed
on the assembler, with a pointer to the current instruction. The
memory that a process can occupy is between a base and limit
value. Instructions must only access memory with a value less than the
limit, but they know nothing about the base value - this is added onto
the memory index provided by the instruction when an instruction is
executed. Each procss also contains a stack and values for all
registers, which are used to store values when the process is
suspended.

\begin{schema}{Processes}
  pids : \power Pid\\
  status : Pid \pfun Status\\
  credits : Pid \pfun \nat\\
  priority : Pid \pfun Priority\\
  instructions : Pid \pfun (\seq Instruction)\\
  inst\_pointer : Pid \pfun \nat_1\\    
  base, limit : Pid \pfun WORD\\
  pregisters : Pid \pfun REGISTERS\\
  pstack : Pid \pfun Stack[WORD]
\where
  pids = \dom(status) = \dom(credits) = \dom(priority) =\\
    \t1 \dom(instructions) = \dom(inst\_pointer) = \dom(base) =\\
    \t1 \dom(limit) = \dom(pstack)\\
  \forall pid : pids @ inst\_pointer(pid) \leq \#(instructions(pid))\\
  \forall pid : pids @ base(pid) + limit(pid) \leq mem\_size
\end{schema}

The $sort$ function takes the credits and priorities of all processes,
and returns a sequence of process IDS sorted firstly by their credits
(the more credits a process has, the higher preference they get), and
if the credits are equal, then their priority. If the priority is
equal, then the order is non-deterministic.

\begin{axdef}
  sort : (Pid \pfun \nat) \cross (Pid \pfun Priority) \pfun \iseq Pid
\where
  sort =
    (\lambda credits : (Pid \pfun \nat); priority :(Pid \pfun Priority) |\\
    \t2 \dom(credits) = \dom(priority) @\\
    \t2 (\mu s : \iseq Pid | \ran(s) = \dom(credits) \land\\
      \t3 (\forall i : 1 \upto \#s - 1 @\\
        \t4 credits(s(i)) > credits(s(i+1)) \lor\\
        \t4 (credits(s(i)) = credits(s(i+1)) \land\\
        \t4 \ priority(s(i)) > priority(s(i)))) @ s))
\end{axdef}

To interrupt a process during execution, the kernel must be in $kernel$ mode.

\begin{zed}
  Mode ::= user | kernel
\end{zed}

For the scheduler, we track which mode the operating system is in, as
well as declaring three ``secondary'' variables, $waiting$, $running$,
and $ready$, to keep the sets of waiting running, and ready variables
respecitvely. In fact, $ready$ is a sequence, and is ordered based on
the credits that each process has. A process with more credits will
have a higher priority. This is fair scheduling, because at each timer
interrupt (the $tick$ operation specified below), the current process
losses one credit, therefore, process spending a lot of time executing
will eventually have a low priority.

\begin{schema}{Scheduler}
  Processes\\
  System\\
  Stack[WORD]\\
  mode : Mode\\
  waiting, running : \power Pid\\
  ready : \iseq Pid
\where
  \# running \leq 1\\
  waiting \cap running \cap \ran ready = \emptyset\\
  waiting \cup running \cup \ran ready = pids\\
  waiting = \{ p : pids | (status \inv) (pWaiting) = p\}\\
  running = \{ p : pids | (status \inv) (pRunning) = p\}\\
  ready = sort((waiting \cup running) \ndres credits, \\
    \t1 (waiting \cup running) \ndres priority)\\
  \forall r : \ran(ready) @ status(r) = pReady\\
  \forall r : running @ credits(r) > 0
\end{schema}

This uses semicolons as conjunctions for predicates, which conforms to
the grammar in the ISO standard, but according to the list of
differences between ZRM and ISO Z on Ian Toyn's website, semicolons
can no longer be used to conjoin predicates.

\begin{schema}{InitScheduler}
    Scheduler\\
    InitStack[WORD]\\
    InitSystem
\where
    pids = \emptyset~; status = \emptyset~; priority = \emptyset\\
    credits = \emptyset~; instructions = \emptyset~; inst\_pointer = \emptyset\\
    waiting = \emptyset~; running = \emptyset~; ready = \langle \rangle\\
    base = \{\}~; limit = \{\}~; pregisters = \{\}\\
    mode = user
\end{schema}

$newProcess$ creates a new process with a unique process ID and a
specified priority, and places this new process on the ready queue.

\begin{schema}{create\_new\_process}
  \Delta Scheduler\\
  \Xi System\\
  priority? : Priority\\
  instructions? : \seq Instruction\\
  base?, limit? : WORD\\
  pid! : Pid
\where
  pid! \notin pids\\
  status' = status \cup \{ pid! \mapsto pReady \}\\
  credits' = credits \cup \{ pid! \mapsto Default\_Credits \}\\
  priority' = priority \cup \{ pid! \mapsto priority? \}\\
  instructions' = instructions \cup \{ pid! \mapsto instructions? \}\\
  inst\_pointer' = inst\_pointer \cup \{ pid! \mapsto 1 \}\\
  base' = base \cup \{ pid! \mapsto base? \}\\
  limit' = limit \cup \{ pid! \mapsto limit? \}\\
  pregisters' =\\
    \t1 pregisters \cup \{ pid! \mapsto \{ r : REGISTER @ \\
      \t2 r \mapsto constant(min(WORD)) \} \}\\
  pstack' =
    pstack \cup \{ pid! \mapsto (\lblot stack == \langle \rangle \rblot) \}\\
    pids' = pids \cup \{pid!\}
\end{schema}

We define a schema that contains only the variables that do not change
when a reschedule occurs.

\begin{zed}
  RescheduleChange == \\
    \t1 Scheduler \hide (status, running, ready, waiting, credits)
\end{zed}

A reschedule occurs when all ready processes have no credits. Every
process, not just the ready processes, have their credits
re-calculated using the formula $credits = credits/2 + priority$. This
implies that the ready process with the highest priority will be the
next process executed.

\begin{schema}{reschedule}
  \Delta Scheduler\\
  \Xi RescheduleChange
\where
  ready \neq \emptyset\\
  \forall r : \ran(ready) @ credits(r) = 0 \implies\\
    \t1 credits' =
      \{ p : pids @ p \mapsto (credits(p) \div 2) + priority(p) \} \land\\
    \t1 status' = status\\
  \lnot (\forall r : \ran(ready) @ credits(r) = 0) \implies\\
    \t1 status' = status \oplus \{ head(ready) \mapsto pRunning \} \land\\ 
    \t1 credits' = credits
\end{schema}

We declare a new schema that contains only the state variables that do
not change when a status change occurs.

\begin{zed}
  StatusChange == Scheduler \hide\\
    \t1 (status, running, waiting, ready, registers, pregisters, pstack)
\end{zed}

Interrupts the currently executing process if the new process is of a
higher priority then the current process and the kernel is in $kernel$
mode.

\begin{schema}{interrupt}
  \Delta Scheduler\\
  \Xi StatusChange\\
  create\_new\_process
\where
  mode = kernel\\
  running = \emptyset\lor(\exists p : running @ priority?\geq priority(p))\\
  \exists r : running @\\
    \t1 status'=status\oplus\{ pid!\mapsto pRunning, r\mapsto pReady \}\land\\
    \t1 pregisters' = pregisters \oplus \{ r \mapsto registers \} \land\\
    \t1 \theta Stack~' = pstack(r)\\
  registers' = pregisters(pid!)
\end{schema}

Remove the currently running process and put it back in the ready queue.

\begin{schema}{remove\_running\_process}
  \Delta Scheduler\\
  \Xi StatusChange
\where
  \exists pid == (\mu r : running) @\\
    \t1 status' = status \oplus \{ pid \mapsto pReady \} \land\\
    \t1 pregisters' = pregisters \oplus \{ pid \mapsto registers \} \land\\
    \t1 pstack' = pstack \oplus \{ pid \mapsto \theta Stack~' \}
\end{schema}

A process becomes blocked if it is waiting on a resource such a an IO
device, or waiting on another process 

\begin{zed}
  block\_process == remove\_running\_process \semi reschedule
\end{zed}

We declare a schema containing only the variables that change for an
unblock.

\begin{zed}
  UnblockProcessChange == Scheduler \hide (status, running, ready, waiting)
\end{zed}

Unblocks a process that is blocked by another process.

\begin{schema}{unblock\_process}
  \Delta Scheduler\\
  \Xi UnblockProcessChange\\
  pid? : Pid\\
\where
  pid? \in pids\\
  status(pid?) = pWaiting\\
  running=\emptyset \iff status'=status \oplus \{ pid? \mapsto pRunning \}\\
  running \neq \emptyset \iff status' = status \oplus \{ pid? \mapsto pReady \}
\end{schema}

Remove a process from the system

\begin{schema}{remove\_process}
  \Delta Scheduler\\
  \Xi Stack[WORD]\\
  \Xi System\\
  pid? : Pid
\where
  pid? \in pids\\
  pids' = pids \setminus \{ pid? \}\\
  status' = \{ pid? \} \ndres status\\
  credits' = \{ pid? \} \ndres credits\\
  priority' = \{ pid? \} \ndres priority\\
  instructions' = \{ pid? \} \ndres instructions\\
  inst\_pointer' = \{ pid? \} \ndres inst\_pointer\\
  base'  = \{ pid? \} \ndres base\\
  limit'  = \{ pid? \} \ndres limit\\
  pregisters' = \{ pid? \} \ndres pregisters\\
  pstack' = \{ pid? \} \ndres pstack
\end{schema}

Update the details in the process table when each instruction is
executed, as well as communicate the current instruction and the base
value for the current process.

\begin{zed}
  ChangeInstPointer == Scheduler \hide (inst\_pointer)
\end{zed}

\begin{schema}{update\_process\_table}
  \Delta Scheduler\\
  inst! : Instruction\\
  base! : WORD
\where
  running \neq \emptyset\\
  (\exists pid == (\mu r : running) @\\
    \t1 inst! = head(instructions(pid)) \land\\
    \t1 base! = base(pid) \land\\
    \t1 (inst\_pointer(pid) = \#(instructions(pid)) \implies\\
      \t2 remove\_process[pid/pid?]) \land\\
    \t1 inst\_pointer(pid) < \#(instructions(pid)) \implies\\
      \t2 inst\_pointer' = \\
        \t3 inst\_pointer \oplus \{ pid \mapsto inst\_pointer(pid) + 1 \})\\
   \theta ChangeInstPointer = \theta ChangeInstPointer~'
\end{schema}

\begin{zed}
  next == exec\_inst \pipe update\_process\_table \semi\\
    \t2 ([\Delta Scheduler | running = \emptyset] \land reschedule) \lor\\
    \t2 ([\Xi Scheduler | running \neq \emptyset])
\end{zed}

\begin{zed}
  idle0 == \lnot \pre next
\end{zed}

\begin{schema}{idle}
  \Xi Scheduler\\
  inst? : Instruction\\
  base? : WORD
\where
  idle0
\end{schema}

\begin{zed}
  tick == next \lor idle
\end{zed}

\begin{zed}
  \vdash? (\forall n : \nat_1 @ n > 0)
\end{zed}
\begin{zed}
  [X] \vdash? \forall x : \power X @ \#x \leq 1 \iff singleton~x
\end{zed}

\end{document}
