%----------------------------------------
\section{Uninterpreted Functions}
%----------------------------------------
In this section we will present the part of our algorithm that is
specific to the theory \euf.  The only mixed atom that is considered
by this theory is $a=b$ where $a$ is $A$-local and $b$ is $B$-local.

\subsection{Leaf Interpolation}

The \euf solver is based on the congruence closure
algorithm~\cite{Detlefs2005}. The theory lemmas are generated from
conflicts involving a single disequality that is in contradiction to a
path of equalities.  Thus, the clause generated from such a conflict
consists of a single equality literal and several disequality
literals.

When computing the partial interpolants of the theory lemmas, we
internally split the mixed literals according to
Section~\ref{sec:purification}.  Then we use an algorithm similar
to~\cite{Fuchs2009} to compute an interpolant.  This algorithm
basically summarises the $A$-equalities that are adjacent on the path
of equalities.  

\ifnewinterpolation
If the theory lemma contains a mixed equality $a=b$ (without negation),
%with auxiliary variable $x_a,x_b$, 
it corresponds to the single
disequality in the conflict.  
This disequality is split into $p_x \xor a=x$ and $\lnot p_x \xor x= b$
and the resulting interpolant depends on the value of $p_x$.  If $p_x=\bot$,
the disequality is part of the $B$-part and $x$ is the
end of an equality path summing up the equalities from $A$.
Thus, the computed interpolant contains a literal of the form $x=s$.
%% This leads to a literal $x_a=s$ in
%% the interpolant.  
If $p_x=\top$, then the $A$-part of the literal is $a \neq x$, and the
resulting interpolant contains the literal $x\neq s$ instead.
Thus, the resulting interpolant can be put into the form $I[p_x \xor x=s]$.
Note that the formula $p_x\xor x=s$ occurs positively in the
interpolant and is the only part of the interpolant containing $x$ and $p_x$.
We define
\begin{align*}
  EQ(x,s) &:= (p_x \xor x = s)
\end{align*}
and require that the partial interpolant of a clause containing the
literal $a=b$ always has the form $I[EQ(x,s)]$ where $x$ and $p_x$ do not
occur anywhere else.

For theory lemmas containing the literal $a\neq b$, the corresponding
auxiliary variable $x$ may appear anywhere in the partial
interpolant, even under a function symbol.  A simple example is the
theory conflict $s\neq f(a) \land a=(x =)b \land f(b)=s$,
which has the partial interpolant $s\neq f(x)$.  In general the 
partial interpolant of such a clause has the form $I(x)$.
\else
If the theory lemma contains a mixed equality $a=b$ (without negation),
%with auxiliary variable $x_a,x_b$, 
it corresponds to the single
disequality in the conflict.  
This disequality is split into $a=x_a$,
$x_a\neq x_b$ and $x_b =b$ and the resulting interpolant depends on
whether we consider the disequality to belong to the $A$-part or to
the $B$-part.  If we consider it to belong to the $B$-part, then $x_a$ is the
end of an equality path summing up the equalities from $A$.
Thus, the computed interpolant has the form $I[x_a=s]$.
%% This leads to a literal $x_a=s$ in
%% the interpolant.  
If we consider $x_a\neq x_b$ to belong to the $A$-part, the
resulting interpolant is $I[x_b\neq s]$.
Note that in both cases the literal $x_a=s$ resp.\ $x_b\neq s$ occurs positively in the
interpolant and is the only literal containing $x_a$ resp.\ $x_b$.
To summarise, the partial interpolant computed for a theory clause
$C\lor a=b$ where $a=b$ has the auxiliary variables $x_a,x_b$ has the form
$I[x_a=s]$ or $I[x_b\neq s]$ and $x_a,x_b$ do not appear at any other
place in $I$.  Both interpolants $I[x_a=s]$ and $I[x_b\neq s]$ are 
partial interpolants of the clause.  From $x_a\neq x_b$ we can derive
the weak interpolant $I[x_b\neq s]$ from the strong interpolant $I[x_a=s]$
using Lemma~\ref{lemma:monotonicity} (monotonicity).
We define
\begin{align*}
  EQ\S(x,s) &:= (x_a = s), &   EQ\W(x,s) &:= (x_b\neq s)
\end{align*}
and label a clause in the proof tree with $I[EQ(x,s)]$ to denote that
the formulae $I[EQ\S(x,s)]$ and $I[EQ\W(x,s)]$ are the strong and weak
partial interpolants.

For theory lemmas containing the literal $a\neq b$, the corresponding
auxiliary variables $x_a,x_b$ may appear anywhere in the partial
interpolant, even under a function symbol.  A simple example is the
theory conflict $s\neq f(a) \land a=(x_a = x_b =)b \land f(b)=s$,
which has the partial interpolants $s\neq f(x_a)$ and $s\neq f(x_b)$
(depending on whether $x_a=x_b$ is considered as $A$- or as
$B$-literal).  We simply label the corresponding theory lemma with the
interpolant $s\neq f(x)$.  In general the label of such a clause has
the form $I(x)$.  The formulae $I(x_a)$ and $I(x_b)$ are the strong
and weak partial interpolants of that clause. Of course, here the
interpolants are equivalent given $x_a=x_b$.
\fi

When two partial interpolants for clauses containing $a=b$ are
combined using~(\ref{rule:res}), i.\,e., the pivot literal is a
non-mixed literal but the mixed literal $a=b$ occurs in $C_1$ and
$C_2$, the resulting partial interpolant may contain $EQ(x,s_1)$ and
$EQ(x,s_2)$ for different shared terms $s_1, s_2$.  In general, we allow
the partial interpolants to have the form $I[EQ(x,s_1)]\dots[EQ(x,s_n)]$.

  
\subsection{Pivoting of Mixed Equalities}

\begin{tacas}
We require that every clause containing $a=b$ with auxiliary variables
$x_a,x_b$ is always labelled with a formula of the form
$I[EQ(x,s_1)]\dots[EQ(x,s_n)]$ and that this is a partial
interpolant of the clause for both $EQ\S$ and $EQ\W$.  As discussed
above, this is automatically the case for the theory lemmas computed
from conflicts in the congruence closure algorithm.  This property is
also preserved by (\ref{rule:res}) and this rule also preserves the
property of being a 
\ifnewinterpolation\else strong or weak \fi
partial interpolant.
\end{tacas}
\begin{techreport}
We require that every clause $C$ containing $a=b$ with auxiliary variables
\ifnewinterpolation $x,p_x$ \else $x_a,x_b$ \fi
is always labelled with a formula of the form
$I[EQ(x,s_1)]\dots[EQ(x,s_n)]$.  
\ifnewinterpolation\else
From this pattern we get the strong
resp.\ weak interpolant by substituting $EQ\S(x,s_i)$ resp.\ $EQ\W(x,s_i)$
($i\in\{1,\dots,n\}$) in $I$.  To show (s-w), assume $\lnot C\proj
A\lor \lnot C\proj B$.
Since $\lnot C$ contains $a \neq b$ we can derive $x_a\neq x_b$.
Then, $EQ\S(x,s_i) \rightarrow EQ\W(x,s_i)$ holds and by
monotonicity we get
\[
I[EQ\S(x,s_1)]\dots[EQ\S(x,s_n)] \rightarrow
I[EQ\W(x,s_1)]\dots[EQ\W(x,s_n)].\]
\fi
%
As discussed above, the partial interpolants computed for
conflicts in the congruence closure algorithm are of the form
$I[EQ(x,s_1)]\dots[EQ(x,s_n)]$. 
This property is also preserved by (\ref{rule:res}), and by Theorem~1 this
rule also preserves the property of being a 
\ifnewinterpolation\else strong or weak \fi
partial interpolant.
\end{techreport}
\ifnewinterpolation\else

\fi
\begin{tacas}
On the other hand, a clause containing the literal $a\neq b$ is
labelled with a formula of the form $I(x)$, i.\,e., the auxiliary
variable $x$ can occur at
arbitrary positions.  Both $I(x_a)$ and $I(x_b)$ are partial
interpolants of the clause.  Again, the form $I(x)$ and the property
of being a partial interpolant is also preserved by (\ref{rule:res}).
\end{tacas}
\begin{techreport}
On the other hand, a clause containing the literal $a\neq b$ is
labelled with a formula of the form $I(x)$, i.\,e., the auxiliary
variable $x$ can occur at 
arbitrary positions.  
\ifnewinterpolation\else
The strong resp.\ weak interpolants are derived
from this pattern as $I(x_a)$ resp.\ $I(x_b)$. 
To show (s-w), assume again $\lnot C\proj A\lor \lnot C\proj B$.
In this case $\lnot C$ contains $a = b$, so we can derive $x_a= x_b$.
Then, $I(x_a) \rightarrow I(x_b)$ holds.
\fi
Again, the form $I(x)$ and the property
of being a partial interpolant is also preserved by (\ref{rule:res}).
\end{techreport}

We use the following rule to interpolate the resolution step on the mixed
literal $a=b$.
%%  At a resolution step where the mixed
%% literal $a=b$ is pivoted, we use the following interpolation rule.
%
\begin{equation}\label{rule:inteq}
\inferrule {C_1\lor a=b:I_1[EQ(x,s_1)]\dots[EQ(x,s_n)] \\ C_2 \lor a\neq b: I_2(x) } 
 {C_1 \lor C_2: I_1[I_2(s_1)]\dots[I_2(s_n)] }
 \tag{rule-eq}
\end{equation}
%
The rule replaces every literal $EQ(x,s_i)$ in $I_1$ with the formula
$I_2(s_i)$, in which every $x$ is substituted by $s_i$. Therefore,
the auxiliary variable introduced for the mixed literal $a=b$ is removed.


\begin{theorem}[Soundness of (\ref{rule:inteq})]
  Let $a=b$ be a mixed literal with auxiliary variable $x$.  If
  $I_1[EQ(x,s_1)]\dots[EQ(x,s_n)]$ 
  \ifnewinterpolation is a partial interpolant 
  \else yields two (strong and weak) partial interpolants \fi
  of $C_1 \lor a=b$ and $I_2(x)$
  \ifnewinterpolation a partial interpolant 
  \else two partial interpolants \fi
  of $C_2 \lor a\neq b$ then
  $I_1[I_2(s_1)]\dots[I_2(s_n)]$ 
  \ifnewinterpolation is a partial interpolant 
  \else yields two partial interpolants \fi
  of the clause $C_1 \vee C_2$.
\end{theorem}
%\begin{tacas}
%The proof is given in the technical report~\cite{atr}.
%\end{tacas}

\begin{techreport}
\begin{proof}
The symbol condition for $I_1[I_2(s_1)]\dots[I_2(s_n)]$ clearly holds
if we assume that it holds for $I_1[EQ(x,s_1)]\dots[EQ(x,s_n)]$ and $I_2(x)$. 
\ifnewinterpolation\else
By construction of weak and strong interpolants (s-w) holds. 
\fi
Hence, after
we show (ind) and (cont), we can apply Lemma~\ref{lem:weakstrongip}.

\ifnewinterpolation
\subsubsection*{Inductivity.}
We assume
\begin{align*}
\forall x,p_x.~ &((p_x \xor a = x) \rightarrow I_1[p_x \xor x=s_1]\dots[p_x \xor x=s_n])\\
&\land (a = x \rightarrow I_2(x))
\end{align*}
and show $I_1[I_2(s_1)]\dots[I_2(s_n)]$.
Instantiating $x:= s_i$ for all $i\in\{1,\dots,n\}$
and taking the second conjunct gives
$\bigwedge_{i\in\{1,\dots,n\}} (a=s_i \rightarrow I_2(s_i))$.
%
Instantiating $p_x:=\bot$ and $x:=a$ and taking the first conjunct gives
$I_1[a=s_1]\dots[a=s_n]$.  
%
With monotonicity we get
$I_1[I_2(s_1)]\dots[I_2(s_n)]$ as desired.


\subsubsection*{Contradiction.}
We have to show
\begin{align*}
I_1[I_2(s_1)]&\ldots[I_2(s_n)] \rightarrow \\
&\exists x,p_x.~(((\lnot p_x \xor x=b) \land I_1[p_x \xor x = s_1]\dots[p_x \xor x = s_n])\\
&\hphantom{\exists x,p_x.~}\lor (x=b \land I_2(x)))
\end{align*}
%
We show the implication for $p_x:=\top$ and $x:=b$.  It simplifies to
\[ I_1[I_2(s_1)]\dots[I_2(s_n)] \rightarrow 
I_1[b \neq s_1]\dots[b \neq s_n] \lor I_2(b)
\]
If $I_2(b)$ holds the implication is true.
If $I_2(b)$ does not hold, we have 
\[\bigwedge_{i\in\{1,\dots,n\}} (I_2(s_i) \rightarrow b \neq s_i)\]
With monotonicity we get $I_1[I_2(s_1)]\dots[I_2(s_n)]
\rightarrow I_1[b\neq s_1]\dots[b\neq s_n]$.
\qed
\else
\subsubsection*{Inductivity.}
We have to show
\begin{align*}
\forall x_a,x_b.~ (a\neq b \proj A \rightarrow I_1\S[x_a=s_1]\dots[x_a=s_n])
&\land (a= b \proj A \rightarrow I_2\S(x_a))\\
& \rightarrow
I_1\S[I_2\S(s_1)]\dots[I_2\S(s_n)]
\end{align*}
Here $I_1\S$ and $I_2\S$ indicates that there may be other patterns unrelated
to the literal $a=b$ that are replaced in the strong interpolant.

Substituting $a=b\proj A \equiv a=x_a\land x_a=x_b$ and instantiating
$x_a=x_b=s_i$ for all $i\in\{1,\dots,n\}$ in the second conjunct gives
$\bigwedge_{i\in\{1,\dots,n\}} (a=s_i \rightarrow I_2\S(s_i))$
%
Substituting $a\neq b\proj A \equiv a=x_a\land x_a \neq x_b$ and instantiating $x_a=a$
and $x_b$ by some value $v$ different\footnote{We assume we always have at least
  two elements in the universe.} from $a$ in the first conjunct gives
$I_1\S[a=s_1]\dots[a=s_n]$.  
%
With monotonicity we get
$I_1\S[I_2\S(s_1)]\dots[I_2\S(s_n)]$ as desired.


\subsubsection*{Contradiction.}
We have to show
\begin{align*}
&I_1\W[I_2\W(s_1)]\dots[I_2\W(s_n)] \rightarrow \\
&\exists x_a,x_b.~((a\neq b \proj B \land I_1\W[x_b\neq s_1]\dots[x_b\neq s_n])
\lor (a= b \proj B \land I_2\W(x_b)))
\end{align*}

If $I_2\W(b)$ holds, instantiate $x_a$ and $x_b$ with $b$.  Then $I_2\W(x_b)$ 
and $a = b\proj B$ hold.  Hence the implication above holds
as desired.

Otherwise, instantiate $x_b$ with $b$ and $x_a$ with some value $v$ different
from $b$.  Then, $a\neq b \proj B$ holds. Since $I_2\W(x_b)$ does not
hold, we have 
\[\bigwedge_{i\in\{1,\dots,n\}} (I_2\W(s_i) \rightarrow x_b \neq s_i)\]
With monotonicity we get 
$I_1\W[x_b\neq s_1]\dots[x_b\neq s_n]$, so the first disjunct holds.
\qed
\fi
\end{proof}
\end{techreport}

%%  LocalWords:  disequality interpolants interpolant EQ inductivity
