\def\vt{\vphantom{t}}

\section{An Example for the Combined Theory}
The previous examples showed how to use our technique to compute an
interpolant in the theory of uninterpreted functions, or the theory of linear
arithmetic. We will now present an example in the combination of these
theories by applying our scheme to a proof of unsatisfiability of the
interpolation problem
\begin{align*}
  A &\equiv t\leq 2a \land 2a \leq s \land f(a)=q\\
  B &\equiv s\leq 2b \land 2b \leq t+1 \land \lnot(f(b)= q)
\end{align*}
where $a$, $b$, $s$, and $t$ are integer constants, $q$ is a constant of the
uninterpreted sort $U$, and $f$ is a function from integer to $U$.

We derive the interpolant using Pudl\'ak's algorithm and the rules shown in this
paper. Note that the formula is already in conjunctive normal form. Since we
use Pudl\'ak's algorithm, every input clause is labelled with $\bot$ if it is
an input clause from $A$, and $\top$ if it is an input clause from $B$.  We
will simplify the interpolants by removing neutral elements of Boolean
connectives.

Since the variables $a$ and $b$ are shared between the theory of uninterpreted
functions and the theory of linear arithmetic, we get some theory combination
clauses for $a$ and $b$.  The only theory combination clause needed to prove
unsatisfiability of $A\land B$ is $a=b\lor\lnot(b\leq a)\lor\lnot(a\leq b)$
which has the partial interpolant $LA(x_1+x_2,0,\ifnewinterpolation F[EQ(x,x_1)]\else EQ(x,x_1)\fi)$\ifnewinterpolation{} where $F[G] \equiv x_1\leq -x_2 \land (x_1\geq -x_2 \rightarrow G)$\fi.  Here, $x_1$ is
used to purify\footnote{Note that we purify the conflict, i.\,e., the negated
  clause} $b\leq a$ and $x_2$ is used to purify $a\leq b$.

We get two lemmas from \laz: The first one, $\lnot(2a\leq s)\lor \lnot(s\leq
2b)\lor a\leq b$, states that we can derive $a\leq b$ from $2a\leq s$ and
$s\leq 2b$. Let $x_3$ be the variable used to purify $\lnot(a\leq b)$.  Note
that we purify the literals in the conflict, i.\,e., the negation of the
lemma. Then, this lemma can be annotated with the partial interpolant
$LA(2x_3-s,-1,\ifnewinterpolation 2x_3 \leq s\else \bot\fi)$.  We can resolve this lemma with the unit clauses from
the input to get $a\leq b$.
\begin{gather*}
\inferrule*
{ 
  \inferrule*[rightskip=\ifnewinterpolation2cm \else .5cm \fi]
  { \lnot(2a\leq s)\lor \lnot(s\leq 2b)\lor a\leq b : LA(2x_3-s,-1,\ifnewinterpolation 2x_3 \leq s\else \bot\fi) \\
    2a\leq s : \bot }
  { \lnot(s\leq 2b)\lor a\leq b : LA(2x_3-s,-1,\ifnewinterpolation 2x_3 \leq s\else \bot\fi) }\\
  s\leq 2b : \top }
{ a\leq b : LA(2x_3-s,-1,\ifnewinterpolation 2x_3 \leq s\else\bot\fi) }
\end{gather*}

The second \laz-lemma, $\lnot(t\leq 2a)\lor\lnot(2b\leq t+1)\lor b\leq a$,
states that we can derive $b\leq a$ from $t\leq 2a$ and $2b\leq t+1$. Let
$x_4$ be the variable used to purify $\lnot(b\leq a)$. Then, we can annotate
the lemma with the partial interpolant $LA(2x_4+t,-1,\ifnewinterpolation 2x_4 + t \leq 0\else\bot\fi)$ and propagate this
partial interpolant to the unit clause $b\leq a$ by resolution with input
clauses.
\begin{gather*}
\inferrule*{
  \inferrule*[rightskip=\ifnewinterpolation 2.5cm \else 1.5cm\fi]
  { \lnot(t\leq 2a)\lor\lnot(2b\leq t+1)\lor b\leq a : LA(2x_4+t,-1,\ifnewinterpolation 2x_4+t \leq 0\else\bot\fi)\quad
    t\leq 2a : \bot }
  { \lnot(2b\leq t+1)\lor b\leq a : LA(2x_4+t,-1,\ifnewinterpolation 2x_4+t \leq 0\else\bot\fi) } \\
   2b\leq t+1 : \top }
{ b\leq a : LA(2x_4+t,-1,\ifnewinterpolation 2x_4+t \leq 0\else\bot\fi) }
\end{gather*}

Additionally, we get one lemma from \euf,
$f(b)=q\lor\lnot(f(a)=q)\lor\lnot(a=b)$, that states that, given $f(a)=q$ and
$a=b$, by congruence, $f(b)=q$ has to hold. Let $x$ be the variable used to
purify $a=b$. Then, we can label this lemma with the partial interpolant
$f(x)=q$. Note that this interpolant has the form $I(x)$ as required by our
interpolation scheme. We propagate this partial interpolant to the unit clause
$\lnot(a=b)$ by resolving the lemma with the input clauses.
\[
\inferrule*
{ 
  \inferrule*[rightskip=.5cm]
  { f(b)=q\lor\lnot(f(a)=q)\lor\lnot(a=b) : f(x)=q \\ f(b)=q : \top}
  { \lnot(f(a)=q)\lor\lnot(a=b) : f(x)=q}\\
  f(a)=q : \bot}
{ \lnot(a=b) : f(x) = q }
\]

From the theory combination clause $a=b\lor
\lnot(b\leq a)\lor\lnot(a\leq b)$ 
and the three unit clauses derived above, we show a contradiction.  
We start by resolving with the unit clause $a=b$ using (\ref{rule:inteq}) and 
produce the partial interpolant $LA(x_1+x_2,0,f(x_1)=q)$.
\[
\inferrule*
{ a=b\lor\lnot(b\leq a)\lor\lnot(a\leq b) : LA(x_1+x_2,0,\ifnewinterpolation F[EQ(x,x_1)] \else EQ(x,x_1)\fi) \\ 
  \lnot(a=b) : f(x) = q}
{ \lnot(b\leq a)\lor\lnot(a\leq b) : LA(x_1+x_2,0,
\ifnewinterpolation F[f(x_1) = q] \else f(x_1) = q\fi) }
\]

The next step resolves on $b\leq a$ using (\ref{rule:intla}).  Note that we
used $x_1$ to purify $b\leq a$ and $x_4$ to purify $\lnot(b\leq a)$.  Hence,
these variables will be removed from the resulting partial interpolant. From
the partial interpolants of the antecedents, $LA(2x_4+t,-1,\ifnewinterpolation 2 x_4 + t \leq 0 \else \bot \fi)$ and
$LA(x_1+x_2,0,\ifnewinterpolation F[f(x_1)=q] \else f(x_1) = q\fi)$, we get the following components:
\begin{align*}
  c_1&=2
  &s_1&=t
  &k_1&=-1
  &F_1(x_4)&\equiv \ifnewinterpolation 2 x_4 + t \leq 0 \else \bot \fi \\
  c_2&=1
  &s_2&=x_2
  &k_2&=0
  &F_2(x_1)&\equiv \ifnewinterpolation F[f(x_1) = q] \else f(x_1) = q\fi
\end{align*}

These components yield $k_3=1\cdot (-1)+2\cdot 0+2\cdot 1=1$.  Furthermore,
$\ceilfrac{k_1+1}{c_1} = 0$ leads to one disjunct in $F_3$. The corresponding
values are $\floorfrac{-t}{2}$, resp.\ $-\floorfrac{-t}{2}$.
\ifnewinterpolation $F_1(\floorfrac{-t}{2})$ is always true and can be omitted.
\fi
The resulting formula $G(x_2) := F_3(\vec x)$ is
\ifnewinterpolation
\begin{align*}
  G(x_2)\equiv{} & -\floorfrac{-t}{2} \leq -x_2\land \left(\floorfrac{-t}{2}\geq -x_2\rightarrow f\left(-\floorfrac{-t}{2}\right)=q\right).
\end{align*}
\else
\begin{align*}
  G(x_2)\equiv{} &t+2\floorfrac{-t}{2}\leq 0\land\left(t+2\floorfrac{-t}{2}\geq 1\rightarrow \bot\right)\land{}\\
  &x_2-\floorfrac{-t}{2}\leq 0\land\left(x_2-\floorfrac{-t}{2}\geq 0\rightarrow f\left(-\floorfrac{-t}{2}\right)=q\right)\\
  {}\equiv{} & x_2-\floorfrac{-t}{2}\leq 0\land \left(x_2-\floorfrac{-t}{2}\geq 0\rightarrow f\left(-\floorfrac{-t}{2}\right)=q\right).
\end{align*}
Note that the first two conjuncts simplify to $\top$ and we remove
them.
\fi
The partial interpolant for the clause $\lnot(a\leq b)$ is
$LA(t+2x_2,1,G(x_2))$.
\[
\inferrule*{
  b\leq a : LA(2x_4+t,-1,\ifnewinterpolation 2x_4 + t \leq 0 \else \bot\fi) \\
  \lnot(b\leq a)\lor\lnot(a\leq b) : LA(x_1+x_2,0,f(x_1) = q) }
{
  \lnot(a\leq b) : LA(t+2x_2,1,G(x_2))
}
\]

In the final resolution step, we resolve $a\leq b$ labelled with partial
interpolant $LA(2x_3-s,-1,\ifnewinterpolation 2 x_3 \leq s \else \bot\fi)$ against $\lnot(a\leq b)$ labelled with
$LA(t+2x_2,1,G(x_2))$. Note that the literals have been purified with $x_3$ and
$x_2$, respectively. We get the components
\begin{align*}
  c_1&=2
  &s_1&=-s
  &k_1&=-1
  &F_1(x_3)&\equiv\ifnewinterpolation 2 x_3 \leq s \else \bot\fi\\
  c_2&=2
  &s_2&=t
  &k_2&=1
  &F_2(x_2)&\equiv G(x_2).
\end{align*}

We get $k_3=2\cdot (-1)+2\cdot 1 + 2\cdot 2=4$.  Again,
$\ceilfrac{k_1+1}{c_1}=0$ yields one disjunct in $F_3$ with the values
$\floorfrac{s}{2}$, and $-\floorfrac{s}{2}$, respectively.  
\ifnewinterpolation Again, $F_1(\floorfrac{s}{2})$ is always true and can be omitted.
\fi
The resulting
formula is
\ifnewinterpolation
\begin{align*}
H\equiv{}& G\left(-\floorfrac{s\vt}{2}\right)\\
{}\equiv{}& -\floorfrac{-t}{2} \leq \floorfrac{s\vt}{2}\land \left(\floorfrac{-t}{2}\geq \floorfrac{s\vt}{2}\rightarrow f\left(-\floorfrac{-t}{2}\right)=q\right).
\end{align*}
\else
\begin{align*}
H\equiv{}&{-s}+2\floorfrac{s\vt}{2}\leq 0\land\left(-s+2\floorfrac{s\vt}{2}\geq 1 \rightarrow\bot\right)\land{}\\
&t-2\floorfrac{s\vt}{2}\leq 0\land\left(t-2\floorfrac{s\vt}{2}\geq -1 \rightarrow G\left(-\floorfrac{s\vt}{2}\right)\right)\\
{}\equiv{}&t-2\floorfrac{s\vt}{2}\leq 0\land\left(t-2\floorfrac{s\vt}{2}\geq -1 \rightarrow -\floorfrac{s\vt}{2}-\floorfrac{-t}{2}\leq 0 \land{} \right.\\
&\quad\left.\left(-\floorfrac{s\vt}{2}-\floorfrac{-t}{2}\geq 0\rightarrow f\left(-\floorfrac{-t}{2}\right)=q\right)\right).
\end{align*}
Again, the first two conjuncts trivially simplify to $\top$ and can be removed.
\fi

The final resolution step yields an interpolant for this problem.
\[
\inferrule{
  a\leq b : LA(2x_3-s,-1,\bot) \\
  \lnot(a\leq b) : LA(t+2x_2,1,G(x_2))
}
{
  \bot : LA(-2s+2t,4,H)
}
\]

\ifnewinterpolation
Thus $H$ is the final interpolant.
\else
We obtain the final interpolant by unfolding the $LA$-form
and some simplifications:
\[
  t\leq s \land \left(t \geq s-2\rightarrow
  \left(t\leq 2\floorfrac{s\vt}{2}\land\left(-\floorfrac{-t}{2}\geq \floorfrac{s\vt}{2}\rightarrow
  f\left(-\floorfrac{-t}{2}\right)=q\right)\right)\right).
\]
\fi
%or with more simplifications and using the identity $\lceil x\rceil = -\floor{-x}$:
%\[
%  t\leq s \land \left(t \geq s-2\rightarrow
%  \left(\frac{t}{2} \leq \floorfrac{s\vt}{2}\land\left(\ceilfrac{t}{2}\geq \floorfrac{s\vt}{2}\rightarrow
%  f\left(\ceilfrac{t}{2}\right)=q\right)\right)\right).
%\]
% 
Now we argue validity of this interpolant.
\paragraph{Interpolant follows from the $A$-part.}
\ifnewinterpolation
The $A$-part contains $2a\leq s$, which implies $a \leq \floorfrac{s}{2}$.
From $t\leq 2a$ we get $-\floorfrac{-t}{2}\leq a$.  Hence,
$-\floorfrac{-t}{2} \leq\floorfrac{s}{2}$.
Moreover, $-\floorfrac{-t}{2}\geq \floorfrac{s}{2}$ implies
$-\floorfrac{-t}{2}=a$.
So with the $A$-part we get $f(-\floorfrac{-t}{2})=q$.
\else
The interpolant expresses that $t\leq s$ holds, which can be deduced from 
the $A$-part.
Moreover from $2a\leq s$, we get $a \leq \floorfrac{s}{2}$.
With $t\leq 2a$ we get $t\leq 2\floorfrac{s}{2}$.  Finally, we show 
$-\floorfrac{-t}{2}\geq \floorfrac{s}{2} \rightarrow f(-\floorfrac{-t}{2})=q$.
Using $-2a \leq -t$, we get $-a \leq \floorfrac{-t}{2}$.
Hence, $a \leq \floorfrac{s}{2} \leq -\floorfrac{-t}{2} \leq a$ implies 
$a=-\floorfrac{-t}{2}$, so with the $A$-part $f(-\floorfrac{-t}{2})=q$
follows.
\fi

\paragraph{Interpolant is inconsistent with the $B$-part.}
The $B$-part implies $s\leq 2b \leq t+1$.  
Hence, we have $\floorfrac{s}{2} \leq b\leq \floorfrac{t+1}{2}$.
A case distinction on whether $t$ is even or odd yields
 $\floorfrac{t+1}{2} = -\floorfrac{-t}{2}$.  Therefore,
$\floorfrac{s}{2} \leq b\leq -\floorfrac{-t}{2}$ holds.
Hence, the interpolant guarantees $f(-\floorfrac{-t}{2})=q$ and
\ifnewinterpolation
$-\floorfrac{-t}{2} \leq \floorfrac{s}{2}$. 
\else
$t\leq 2\floorfrac{s}{2}$. The latter implies
$-\floorfrac{-t}{2} \leq \floorfrac{s}{2}$. 
\fi
Hence, $b=-\floorfrac{-t}{2}$
and with $f(b)\neq q$ from the $B$-part we get a contradiction.

\paragraph{Symbol condition is satisfied.}
The symbol condition is trivially satisfied since $\symb(A)=\{a,t,s,f,q\}$ and
$\symb(B)=\{b,t,s,f,q\}$.  The shared symbols are $t$, $s$, $f$, and $q$ which
are exactly the symbols occurring in the interpolant.

%Since the interpolant follows from the $A$-part, is inconsistent with the
%$B$-part, and only contains shared symbols, it is a valid interpolant for the
%interpolation problem $A$ and $B$.

\ifnewinterpolation\else
A close inspection of the last proof reveals that $H$ is already a valid
interpolant of $A$ and $B$. 
This shows that in a certain sense the produced interpolants are not minimal.
It may be useful to investigate more closely which parts can be safely
omitted.
\fi

