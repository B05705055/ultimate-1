%\documentclass[english]{scrartcl}
\documentclass{llncs}
%\usepackage[T1]{fontenc}
%\usepackage[latin9]{inputenc}
\setlength{\parskip}{\smallskipamount}
\setlength{\parindent}{0pt}
%\usepackage{units}
%\usepackage{amsthm}
\usepackage{amsmath}
\usepackage{amssymb}

\newcommand{\ceilfrac}[2]{\left\lceil \frac{#1}{#2} \right\rceil}
\newcommand{\floorfrac}[2]{\left\lfloor \frac{#1}{#2} \right\rfloor}
\newcommand{\proj}[2]{#1\downharpoonright #2}

\usepackage{mathpartir}

\usepackage[pdftex, bookmarks=true, bookmarksopen=true]{hyperref}


\newcommand{\sod}{s_1\negthinspace\downarrow}
\newcommand{\stu}{s_2\negthinspace\uparrow}


\title{Interpolation of Mixed Literals for the Theories of 
Linear Arithmetic (Integer and Real)  and Uninterpreted Functions}

\author{J\"urgen Christ, Jochen Hoenicke, Alexander Nutz}

%\institute{Chair of Software Engineering, Uni Freiburg}

\setcounter{tocdepth}{3} %useful (only) for hyperref bookmarks

\begin{document}

\maketitle

%\tableofcontents

%----------------------------------------
\section{Introduction}
%----------------------------------------

Motivate Craig Interpolation.
McMillan/Pudl\'ak algorithm.
Nelson-Oppen theory combination.
problems with mixed Literals.


%----------------------------------------
\section{Related Work}
%----------------------------------------

%----------------------------------------
\section{Preliminaries}
%----------------------------------------

craig interpolants

partial interpolants 

(purification)

% Given an interpolation problem $F_1,\dots,F_n$, we define for a
% literal $\ell$ and every partition $i$ a projection $\proj{\ell}{i}$ with
% the following porperties:
% \begin{itemize}
% \item $\ell \equiv \exists y_{\ell} (\proj{\ell}{1} \land \dots \land \proj{\ell}{n})$,
% \item $symbols(\proj{\ell}{i}) \subseteq symbols(F_i)$,
% \item $vars(\proj{\ell}{i}) \subseteq \{ y_{\ell}\}$.
% \end{itemize}
% Here $y_{\ell}$ is a fresh variable that may be used if the literal
% contains symbols from different partitions.
% 
% 
% Examples Given the formulas $F_1 \equiv a=s \land s=t$ and $F_2\equiv
% t=b \land s\neq b$.  We can define:
% 
% \[
% \begin{array}{cc}
% \proj{a=s}{1} \equiv a=s & \proj{a=s}{2} \equiv \bot \\
% \proj{b\neq s}{1} \equiv \bot  & \proj{b\neq s}{2} \equiv b\neq s \\
% \proj{s = t}{1} \equiv s=t & \proj{s=t}{2} \equiv s=t \\
% \proj{a= b}{1} \equiv a=y_{a=b}  & \proj{a= b}{2} \equiv y_{a=b}= b \\
% \proj{a\neq b}{1} \equiv a=y_{a=b}  & \proj{a\neq b}{2} \equiv y_{a=b}\neq b
% \end{array}
% \]
% 
% 
% $x_1 + x_2 + x_3 \leq 5$ is projected to
% 
% $x_1 = y_{\ell,1}$, $y_{\ell,1}+x_2 = y_{ell,2}$ and
% $y_{\ell,2} + x_3 \leq 5$.
% 
%  If $s$ and $t$ are contained
% in $F_1$ and $F_2$
% 
% A conflict $C$ for an interpolation problem $F_1,\dots,F_n$ is a
% conjunction of literals that is unsatisfiable in the context
% $F_1,\dots,F_n$.  Every clause in the proof tree is the negation of a
% conflict.  We define a projection $\proj{C}{i}$ as the conjunction of the projection of the literals, i.e.
% \[\proj{(\ell_1\land \dots\land\ell_m)}{i} = 
% \proj{\ell_1}{i} \land \dots\land  \proj{\ell_m}{i}\]
% 
% 
% 
% 
% \begin{definition}[Partial Interpolants]
% Given an interpolation problem $F_1,\dots, F_n$.  A sequence of
% partial interpolants $I_0,I_1,\dots,I_n$ for a conflict $C$ has the
% following properties:
% \begin{itemize}
% \item $I_0 = \top$ and $I_n= \bot$
% \item $I_{i-1} \land F_i \land (\proj{C}{i}) \Rightarrow I_i$
% \end{itemize}
% \end{definition}

\subsubsection{Restriction to A/B}

For computing the partial interpolants we need a notion of what $C|A$ and $C|B$ 
are for any clause $C$. Normally we can just divide the literals accordingly: 
Literal with A-local symbols belong to $C|A$, and analogously for $C|B$, by 
convention literals with only shared symbols belong to $C|B$. But what if a 
theory introduces mixed literals like $a=b$ where $a$ is A-local and $b$ is 
B-local?

We replace ab-mixed literals by introducing a fresh auxiliary variable which we 
denote by $x$ in the following. We count this variable as shared between A and B. 
It will be removed from the partial interpolants by a special interpolation rule
at the node in the proof tree where the mixed literal is pivoted by resolution. 
Rules for the different theories will be introduced in the following sections.

To remove a mixed literal $\ell$ from a conflict, we replace it by the 
equisatisfiable formula $\ell|A \wedge \ell|B$.

Suppose we have a mixed literal $a=b$:

\begin{align*}
(a=b)|A & := (a=x) \\
(a=b)|B & := (x=b) \\
(a\neq b)|A & := (a=x) \\
(a\neq b)|B & := (x\neq b)
\end{align*}

Suppose we have a mixed literal $a\leq b$:

\begin{align*}
(a\leq b)|A & := (a\leq x) \\
(a\leq b)|B & := (x\leq b) \\
(a > b)|A & := (a\geq x) \\
(a > b)|B & := (x > b)
\end{align*}

For this purification to be correct we must be sure that for any formulae $F,A,B$: 
$F \leftrightarrow \exists x. F|A \wedge F|B$. This is clearly the case.

\subsubsection{Some Lemmas}

The following lemmas will help us prove the correctness of our proposed new 
interpolation rules.

\paragraph{Notation} In the following we denote by $F[\phi]$ a formula 
in negation normal form (NNF) with a placeholder $\phi$ that occurs
positive in the formula.  Substituting this placeholder by a formula
$G$ is denoted by $F[G]$.  By $F(x)$ we denote a formula with a
placeholder $x$ that stands for an arbitrary subterm of some literal
in $F(x)$.  The substitution of $x$ with a term $t$ is denoted by $F(t)$.

\begin{lemma}[monotonicity]
  Given a formula $F[\phi]$ in negation normal form with a placeholder
  formula $\phi$ occuring only positive in the formula.  Then
  \[ G_1 \rightarrow G_2 \implies F[G_1] \rightarrow F[G_2] \]
\end{lemma}
\begin{proof}
  Induction over the number of logical connectives in $F[\phi]$.  The
  base case $F[\phi] \equiv \phi$ is trivial.  For the induction step
  observe that if $F_1[G_1]\rightarrow  F_1[G_2]$ and
  $F_2[G_1]\rightarrow  F_2[G_2]$, then 
  \[F_1[G_1] \land F_2[G_1] \rightarrow F_1[G_2] \land F_2[G_2] 
  \text{ and }
  F_1[G_1] \lor F_2[G_1] \rightarrow F_1[G_2] \lor F_2[G_2]\]
\end{proof}

\begin{lemma}[deep substitution]
  Let $F_1[\phi]$ and $F_2[\phi]$ be two formulas with the place holder
  formula $\phi$.  
  
  1. If $G_1\land G_2\rightarrow G_3$ holds, then

  \[ F_1[G_1] \land F_2[G_2] \rightarrow F_1[F_2[G_3]]. \]

  2. If $G_3 \rightarrow G_1\lor G_2$ holds, then
  \[ F_1[F_2[G_3]] \rightarrow F_1[G_1] \lor F_2[G_2]. \]

  3. If $G_3 \rightarrow G_1\land G_2$ holds, then
  \[ F_1[F_2[G_3]] \rightarrow (F_1[G_1] \land F_2[G_2]) \lor F_1[\bot] \lor F_2[\bot]. \]  
\end{lemma}
\begin{proof}
  \begin{align*}
  1.\qquad &\quad
    (G_1 \land G_2) \rightarrow G_3)\\
    \Leftrightarrow&\quad
    G_1 \rightarrow (G_2 \rightarrow G_3)\\
    \text{\{monotonicity\}}\Rightarrow&\quad 
    G_1 \rightarrow (F_2[G_2] \rightarrow F_2[G_3])\\
    \Leftrightarrow&\quad
    F_2[G_2] \rightarrow (G_1 \rightarrow F_2[G_3])\\
    \text{\{monotonicity\}}\Rightarrow&\quad 
    F_2[G_2] \rightarrow (F_1[G_1] \rightarrow F_1[F_2[G_3]])\\
    \Leftrightarrow&\quad
    F_1[G_1] \land F_2[G_2] \rightarrow F_1[F_2[G_3]]\\
  \end{align*}

  \begin{align*}
    2.\qquad &\quad
    (G_3 \rightarrow G_1\lor G_2)\\
    \Leftrightarrow&\quad
    \lnot G_1 \rightarrow (G_3 \rightarrow G_2)\\
    \text{\{monotonicity\}}\Rightarrow&\quad 
    \lnot G_1 \rightarrow (F_2[G_3] \rightarrow F_2[G_2])\\
    \Leftrightarrow&\quad
    \lnot F_2[G_2] \rightarrow (F_2[G_3] \rightarrow G_1)\\
    \text{\{monotonicity\}}\Rightarrow&\quad 
    \lnot F_2[G_2] \rightarrow (F_1[F_2[G_3]] \rightarrow F_1[G_1])\\
    \Leftrightarrow&\quad
    F_1[F_2[G_3]] \rightarrow F_1[G_1] \lor F_2[G_2] \\
  \end{align*}

  3. From $G_3\rightarrow G_1\land G_2$ we get 
  $G_3\rightarrow G_1 \lor \bot$ and $G_3 \rightarrow \bot \lor G_2$.  Hence by 2.:
  \begin{align*}
    F_1[F_2[G_3]] \rightarrow& F_1[G_1] \lor F_2[\bot]\\
    F_1[F_2[G_3]] \rightarrow& F_1[\bot] \lor F_2[G_2]
  \end{align*}

  Combining these results, we get
  \begin{align*}
    F_1[F_2[G_3]] &\rightarrow
    (F_1[G_1] \lor F_2[\bot]) \land (F_1[\bot] \lor F_2[G_2])\\
  &\leftrightarrow
  ((F_1[G_1] \lor F_2[\bot]) \land F_1[\bot]) \lor
  (F_1[G_1]\land F_2[G_2]) \lor (F_2[\bot] \land F_2[G_2]) \\
  &\rightarrow
  (F_1[G_1]\land F_2[G_2]) \lor F_1[\bot] \lor F_2[\bot] 
  \end{align*}
\end{proof}

%----------------------------------------
\section{Uninterpreted Functions}
%----------------------------------------

\subsection{Leaf Interpolation}

\begin{itemize}
 \item for reasoning over UF we use DAG-based approach
 \item  for interpolation we use a cycle in the cc-graph that contains exactly one 
 inequality
 (there must be at least one such cycle as we need one for proving unsat)
 \item  the normal procedure generates an interpolant by summarizing the ``red'' edges
 in the cycle
  problem: an edge between an a-local and a b-local term does not have a color.
   thus we need mixed interpolation when the cycle contains such an edge 
 \item  two cases: the edge is either an equality, or a disequality
 \begin{itemize}
  \item if it is a disequality we insert a fresh auxiliary variable, say $x$, which is 
  considered as shared, (it will be removed when the literal is pivoted, thus it does
  not occur in the final interpolant) like this:
  $a=x\neq b$, the $a=x$-edge becomes red, the other one blue, we can interpolate as 
  normal. 
  Our interpolant will definitely contain the summary of $a=x=...=s$ for some other 
  DAG-node $s$, this summary is $a=s$. (this will become important for the 
  interpolation rule later which removes the $x$)
  (in the disequality-case, $a\neq b$ can not influence the interpolant in any other 
  way because disequalities only introduce one edge each, in contrast to equalities 
  which may introduce more edges by function congruence)
  \item if it is an equality, it may occur in the cycle ``as its self'', or it may have
  created mixed edges by function congruence, i.e. in the cycle there may be an edge
  $f(..,a,..)=f(..,b,..)$. This edge does not have a color either. Thus we purify
  $a=b$ to $a=x=b$ (even if $a=b$ is not in the cycle), lift the function application we 
  need, and insert the congruence edges that follow from the new equalities (it suffices to
  do the ones from the cycle)
  Thus we get $f(..,a,..)=f(..,x,..)=f(..,b,..)$ over which we can interpolate as usual.
 \end{itemize}
\end{itemize}
  
  
\subsection{Pivoting of Mixed Literals}

\[\inferrule {a=b \vee C_1:I_1[x=s] \\ a \neq b \vee C_2: I_2(x) } {C_1 \vee C_2: I_1[I_2(s)] } \]

\begin{theorem}
  $I_1[I_2(s)]$ is a partial interpolant of $A$ and $B$ at clause $C_1 \vee C_2$.
\end{theorem}

\begin{proof}

First we need to show \[A \wedge \neg C_1|A \wedge \neg C_2|A \stackrel{!}{\models} I_1[I_2(s)]\]

By assumption $I_1[x=s]$ is a partial interpolant, at clause $C_1 \vee a=b$:
\begin{align*}
  & A \wedge \neg C_1|A \wedge a\neq b|A \models I_1[x=s] & (1)
\end{align*}
Also by assumption $I_2(s)$ is a partial interpolant, at clause $C_2 \vee a \neq b$:
\begin{align*}
  & A \wedge \neg C_2|A \wedge a=b|A \models I_2(x) \quad & (2)
\end{align*}
We assume $A \wedge \neg C_1|A \wedge \neg C_2|A$, the antecedent of our proof goal, so $I_1[I_2(s)]$ 
remains to be shown.

$(1)$ and the previously defined restricion rules give us:
\begin{align*}
  & a=x \models I_1[x=s]  \\
  \Rightarrow & \models I_1[a=s] (*)
\end{align*}

$(2)$ gives us:
\begin{align*}
  & a=x \models I_2(x) \\
  \text{\{inst x:=s\}} & \Rightarrow a=s \models I_2(s) \\
  & \Rightarrow I_1[a=s] \models I_1[I_2(s)] \\
  (*) & \Rightarrow \models I_1[I_2(s)]
\end{align*}

Second we need to show \[B \wedge \neg C_1|B \wedge \neg C_2|B \wedge I_1[I_2(s)] \stackrel{!}{\models} \bot\]

Assume $B \wedge \neg C_1|B \wedge \neg C_2|B$. Then our proof goal is $I_1[I_2(s)] \models \bot$.

  $I_1[x=s]$ is a partial interpolant, at clause $C_1 \vee a=b$, thus (for any x):

\begin{align*}
  & x\neq b \wedge I_1[x=s] \models  \bot (1)\\
  \text{\{inst x:=s\}} \Rightarrow & s \neq b \wedge I_1[\top] \models \bot (1')
\end{align*}

$I_2(s)$ is a partial interpolant, at clause $C_2 \vee a\neq b$, thus:

\begin{align*}
  & x=b \wedge I_2(x) \models  \bot (2)\\
  \text{\{inst x:=s\}} \Rightarrow & s=b \wedge I_2(s) \models \bot (2')\\
\end{align*}

Case $s\neq b$:
\begin{align*}
  & I_2(s) \models \top \\
  \text{\{monotonicity\}} & I_1[I_2(s)] \models I_1[\top] \\
  (1'), case & I_1[I_2(s)] \models \bot \\
\end{align*}

Case $s=b$
\begin{align*}
  (2) & I_2(s) \models \bot & \\
  \text{\{monotonicity\}} & I_1[I_2(s)] \models I_1[\bot] & (*) \\
  \text{\{in (1) choose $x\neq s$ \footnote{Here we need our assumption that the domain has at least two elements.}\}}  & I_1[\bot] \models \bot & \\
  \text{\{$(*)$, transitivity\}} & I_1[I_2(s)] \models \bot & \\
\end{align*}
\qed 
\end{proof}


%----------------------------------------
\section{Linear Real and Integer Arithmetic}
%----------------------------------------


The interpolants for mixed inequalities contain subformulas of the
form $EQ(s(\vec x), k, F(\vec x))$, where $k\in
\mathbb{Q}$ is a constant, $F$ an arbitrary formula, and $s(\vec
x_+,\vec x_-) = \sum c_i x_i + t$ is an affine
term, $c_i \neq 0$ and $t$ does not depend on $\vec x$.  We denote
by $\vec x_+$ the variables in $x_i \in \vec x$ with $c_i > 0$ and
by $\vec x_-$ the variables in $x_i \in \vec x$ with $c_i < 0$.

The formula $EQ(s,k,F)$ stands for two different interpolants, one
strong and one weak.   These are:
\begin{align*}
EQ_S\left(s(\vec x), k, F(\vec x)\right) :\equiv & 
  s(\vec x) \leq 0 \land 
  \forall \vec x'_+ \leq \vec x_+, \vec x'_-\geq \vec x_-: 
	  s(\vec x') < -k \vee F(\vec x')\\
EQ_{W} \left(s(\vec x),k,F(\vec x)\right) :\equiv & 
  s(\vec x) < -k \vee F(\vec x)
\end{align*}
Having two interpolants gives more flexibility.  The strong
interpolant allows for changing the free variables in $F$, provided
that we change them only in the direction that will decrease $s$.
Obviously, the strong interpolant $EQ_S(s,k,F)$ implies the weak
interpolant $EQ_W(s,k,F)$ but not vice versa.  Having quantifiers in
the strong interpolant is no problem; in the end, we return
the weak interpolant. We will show that the strong interpolant follows
from the strong interpolant of the parent proof nodes or the $A$ part
for leaves and that the weak interpolant implies the weak
interpolants of the parent proof nodes, or the negation of the $B$
part for leaves.

The number $k$ in $EQ(s,k,F)$ can either be $-\varepsilon$, or is an
epsilon-free term $k \in \mathbb{Q}$.  If $k < 0$, then
$F = \bot$ must hold.

\subsection{Leaf Interpolation}

\subsection{Pivoting of Mixed Literals}

We will first define how to merge the $EQ$ formulas when pivoting on a
mixed literal.  Let $y$ be the variable introduced by the mixed
literal, then the interpolant of the first clause may only contain $y$
with positive coefficient, i.e. $EQ(c_1 y + s_1(\vec x_1), k_1, F_1(y,
\vec x_1))$, and the interpolant of the second clause may contain $y$
only in subformulas of the form $EQ(-c_2 y + s_2(\vec x_1), k_2,
F_1(y, \vec x_2))$.  

We combine these two formulas to a single formula $EQ(s, k, F_3)$ with
$s := \frac{s_1}{c_1} + \frac{s_2}{c_2}$.  We combine the variables
(except $y$), i.\,e., $\vec x = \vec x_1 \cup \vec x_2$.  Note that
$\vec x_{1+} \cap \vec x_{2-} = \emptyset = \vec x_{2+} \cap \vec
x_{1-}$.  Otherwise the resolvent would contain the corresponding
literal both positively and negatively which is not possible in a
well-formed proof tree.

If $y$ is an integer variable, then 
$k_3 = \frac{k_1}{c_1} + \frac{k_2}{c_2} + 1$.  

For a real variable, we set $k_3 = \frac{k_1}{c_1} + \frac{k_2}{c_2}$;
however, if $k_1=k_2=-\varepsilon$ then $k_3 = -\varepsilon$ and if
$k_{i}\in\mathbb{Q}, k_{\frac{2}{i}}=-\varepsilon, i=1,2$, then $k_3 =
\frac{k_i}{c_i}$.  Finally we compute $F_3$ as follows

\begin{align*}
  &&\textnormal{(int case)} \\F_3(\vec x) & :\equiv 
  \bigvee_{i=0}^{\ceilfrac{k_1+1}{c_1}}
  (EQ^1_W\left(\ceilfrac{-s_1(\vec x_1) - i c_1}{c_1}\right)
    \land EQ^2_W\left(\ceilfrac{-s_1(\vec x_1) - i c_1}{c_1}\right)\\
&&\textnormal{(real case)} \\F_3(\vec x) & :\equiv
  EQ^1_W\left(\frac{-s_1(\vec x)}{c_1}\right)
  \land EQ^2_W\left(\frac{-s_1(\vec x)}{c_1}\right)\\
\end{align*}

Note that the formula is asymmetric.  If $\ceilfrac{k_2+1}{c_2} <
\ceilfrac{k_1+1}{c_1}$ we can replace $-s_1$ by $s_2$, $k_1$ by $k_2$,
and $c_1$ by $c_2$ and change ceiling to flooring.  In
the real case we require that $k_1 \leq 0$.  If both $k_1$ and $k_2$
are positive (which can happen only for LIRA logics that mix integer
and real arithmetic), our algorithm will fail.


\begin{lemma}
  For all $s(\vec x_+, \vec x_-),k,F(\vec x_+,\vec x_-)$:
\[  EQ_S(s(\vec x_+,\vec x_-),k,F(\vec x_+,\vec x_-)) \Rightarrow EQ_W(s(\vec x_+,\vec x_-),k,F(\vec x_+,\vec x_-))\]
\end{lemma}
\begin{proof}
  Instantiate the vector $\vec x'_+,\vec x'_-$ in $EQ_{S}$ with $\vec x_+,\vec x_-$.
\end{proof}


\subsubsection{Pivoting Rule}
The pivoting rule for pivot element $t\leq0$ with auxiliary variable $y$ is:

\[\inferrule 
{t\leq 0 \vee C_1 : I_1[EQ{(c_1y + s_1, k_1, F_1(y))}] \\
t > 0 \vee C_2: I_2[EQ{(-c_2y + s_2, k_2, F_2(y))}]} 
{C_1 \vee C_2: I_1[I_2[EQ(s_1/c_1 + s_2/c_2, k, F_3)] } 
\]

with $k$ and $F_3$ as defined above.



\subsection{proof of correctness}

Shorthands: 

For later reference we write out all the interpolants that occur in the rule as follows:\\


\begin{align*}
EQ_S^1(y) & := \underbrace{c_{1}y+s_{1}(\vec x) \leq 0}_{(1.1)} \wedge 
( \underbrace{ \forall y'\leq y,\vec x'_+\leq \vec x_+, \vec x'_-\geq \vec x_-: -k_1 \leq c_{1}y'+s_{1}(\vec x')
\rightarrow F_1 (y',\vec x')}_{(1.2)} ) \\
EQ_S^2(y) & := \underbrace{-c_2 y+s_2(\vec x) \leq 0}_{(2.1)} \wedge 
( \underbrace{  \forall y'\geq y,\vec x'_+\leq \vec x_+, \vec x'_- \geq \vec x_-: -k_2 \leq -c_{2}y'+s_{2}(\vec x')
\rightarrow F_2 (y', \vec x')}_{(2.2)} ) \\
EQ_S^3 & := \underbrace{s_1(\vec x)/c_1 + s_2(\vec x)/c_2 \leq 0}_{(3.1)} \wedge 
( \underbrace{ \forall \vec x'_+\leq \vec x_+, \vec x'_- \geq \vec x'_+: -k_3 \leq s_2(\vec x')/c_2 + s_1(\vec x')/c_1
\rightarrow F_3(\vec x')}_{(3.2)} \\
& & \\
EQ_W^1(y) & := 
( \underbrace{ - k_1 \leq c_{1}y+s_{1}(\vec x)}_{(1.3)} \rightarrow
\underbrace{F_1 (y, \vec x)}_{(1.4)} ) \\
EQ_W^2(y) & := 
( \underbrace{- k_2 \leq -c_{2}y+s_{2}(\vec x)}_{(2.3)} \rightarrow
\underbrace{F_2 (y, \vec x))}_{(2.4)} ) \\
EQ_W^3 & := 
( \underbrace{ - k_3 \leq s_1(\vec x)/c_1 + s_2(\vec x)/c_2}_{(3.3)} \rightarrow 
\underbrace{F_3(\vec x)}_{(3.4)} ) \\
\end{align*}

\begin{lemma}\label{lemma_la1}
$\exists y. EQ_S^1(y) \wedge EQ_S^2(y) \rightarrow EQ_S^3$
\end{lemma}

\begin{proof}

(integer case)

Fix $y$ such that $EQ_S^1(y)$ and $EQ_S^2(y)$
From (1.1) and (2.1) it follows that 
  \[\frac{s_2(\vec x)}{c_2} \leq y \leq -\frac{s_1(\vec x)}{c_1}.\] 
By transitivity and some simple transformations we get (3.1).

Now we show that (3.2) holds for all $\vec x'_+ \leq \vec x_+$ and
all $\vec x'_- \leq \vec x_-$.  Instantiate (1.2) and (2.2) with the
same $\vec x'$.  
We show that $F_3(\vec x')$ holds using a case distinction on $y$.

\paragraph{Case $y < \floorfrac{-s_1(\vec x')-c_1\ceilfrac{k_1+1}{c_1}}{c_1}$:}
%  Since $y$ occurs negatively in $-c_2y-s_2$, 
Instantiate $y'$ in (2.2) with 
$\floorfrac{-s_1(\vec x')-c_1\ceilfrac{k_1+1}{c_1}}{c_1}$ 
(which we can do because the universal quantifier in (2.2) ranges over all $y'\geq y$)
and obtain
$EQ^2_W(\floorfrac{-s_1(\vec x')-c_1\ceilfrac{k_1+1}{c_1}}{c_1})$, which is part of the
last disjunct in $F_3(x)$ where $i=\ceilfrac{k_1+1}{c_1}$.

Also $EQ^1_W(y')$, which is the other part of the last
disjunct of $F_3(x)$, holds for the same $y'$ since $s_1(\vec x') + c_1 y' < -k_1$:

\begin{align*}
  & \phantom{\Rightarrow} s_1(\vec x') + c_1 y' < -k_1 \\
  & \Leftrightarrow y' < \frac{-s_1(\vec x') -k_1}{c_1} \\
  & \Leftrightarrow 
    \floorfrac{-s_1(\vec x')-c_1\ceilfrac{k_1+1}{c_1}}{c_1} < \frac{-s_1(\vec x') -k_1}{c_1}
\end{align*}

Hence $F_3$ holds.

\paragraph{Case $\floorfrac{-s_1(\vec x')-c_1\ceilfrac{k_1+1}{c_1}}{c_1}\leq y$:}
%  From (1.1) we know $y\leq \floorfrac{-s_1(\vec x)}{c_1}$.
We can instantiate $y'$ in both (1.2) and (2.2) with $y$ (and $x'$ with $x$). 

The resulting terms are
identical with  $EQ^1_W(y)$ and $EQ^2_W(y)$.  Since $-s_1(\vec x) \leq
-s_1(\vec x')$, there is an $i$ with $0\leq i \leq
\ceilfrac{k_1+1}{c_1}$ such that $y = \floorfrac{-s_1(\vec
  x')-ic_1}{c_1}$, so the corresponding disjunct in $F_3(\vec x)$ holds.  
%    Hence (3.2) and thus $EQ^3_S$ holds.
  \qed
\end{proof}

\begin{proof}
(real case)

As in the integer case we can prove (3.1) from (1.1) and (2.1), and on the way we get:

\[\frac{s_2(x)}{c_2} \leq y\leq \frac{-s_1(x)}{c_1} (*)\]

Because the universal quantifier in (2.2) ranges over all $y'\geq y$, we can choose 
$y'=\frac{-s_1(x)}{c_1}$ and we get that $EQ^2_W(y')$ holds. This is the right conjunct
in $F_3(x)$.
  
For showing (3.2) it suffices to show $F_3(x)$ under the condition that the antecedent 
$-k_3\leq \frac{s_1(x)}{c_1} + \frac{s_2(x)}{c_2}$ holds. 
Together with (3.1) the antecedent of (3.2) gives us:
\begin{align*}
  & -k_3  \leq \frac{s_1(x)}{c_1} + \frac{s_2(x)}{c_2} \leq 0 \\
  \text{\{$k_3 \leq 0$\}} & \Rightarrow  k_3 = 0 \wedge \frac{s_1(x)}{c_1} + \frac{s_2(x)}{c_2}=0 \\
  & \Rightarrow \frac{s_2(x)}{c_2} = - \frac{s_1(x)}{c_1} \\
  \text{\{$(*)$\}} & \Rightarrow  y = - \frac{s_1(x)}{c_1}
\end{align*}

Because $y=\frac{-s_1(x)}{c_1}$, we can instantiate $y'$ with it in (1.1) and obtain that 
$EQ^2_W(y')$ holds. This is the other conjunct in $F_3$.

\qed

%  If $y$ is a real variable, the proof works analogously.  One needs to
%  omit the flooring and ceiling operators.  Moreover, we use
%  $\varepsilon$ to get the smallest real number greater than $s_2+k_2$.
%  If $k_2$ is $-\varepsilon$, this will just remove the $\varepsilon$.
%  Otherwise the $\varepsilon$ may change the outcome of ceiling and
%  flooring operations hidden inside $F_1$ and $F_2$.
\end{proof}



\begin{theorem}
$I_1[I_2[EQ(s_1/c_1 + s_2/c_2, k, F_3)]$ has the inductive property of a partial interpolant
of $A$ and $B$, at clause $C_1 \vee C_2$. 
\end{theorem}
\begin{proof}
To show:
\begin{align*}
A\wedge\neg C_{1}\backslash B\wedge\neg C_{2}\backslash B 
& \models I_1[I_2[EQ(c_2 s_1 + c_1 s_2, k_3, F_3)] \label{eq:15}
\end{align*}

In the following assume that $A\wedge\neg C_{1}\backslash B\wedge\neg C_{2}\backslash B$ holds,
we have to show $I_1[I_2[EQ_S^3]]$.

$I_1[EQ_S^1(y)]$ and $I_2[EQ_S^2(y)]$ are partial interpolants of $A$ and $B$ at clauses
$(C_1 \vee a - y \leq 0)$ and $(C_2 \vee a - y > 0)$ respectively. So given the assumption above
we know:
\begin{align*}
a - y \geq 0 & \models I_1[EQ_S^1(y)] \\
a - y \leq 0 & \models I_2[EQ_S^2(y)]
\end{align*}

Thus, in our context, $\models I_1[EQ_S^1(t_A)]$ and $\models I_2[EQ_S^2(t_A)]$ hold.

By Lemma \ref{lemma_la1} we know that $\exists y. EQ_S^1(y) \wedge EQ_S^2(y) \rightarrow EQ_S^3$.

With the first part of the deep substitution lemma, we obtain that our proof goal is implied by what we already know:
\begin{align*}
  &  & I_{2}\left[EQ_S^2\left(t_{A}\right)\right]\wedge I_{1}\left[EQ_S^1\left(t_{A}\right)\right]\rightarrow I_{1}\left[I_{2}\left[EQ_S^3\right]\right]
\end{align*}

% As we have shown that 
% $I_{2}\left[EQ_{R}^{\left(2\right)}\left(t_{A}\right)\right]\wedge I_{1}\left[EQ_{R}^{\left(1\right)}\left(t_{A}\right)\right]$
% holds, we are done.
\end{proof}

\begin{lemma} \label{lemma_la2}
$EQ_W^3 \rightarrow \exists y. (EQ_W^1(y) \wedge EQ_W^2(y))$
\end{lemma}
\begin{proof}
(integer case)

Case (3.3) holds:
Then (3.4), that is $F_3(x)$, has to hold which immediately implies that there is one $y$ fulfilling
$EQ^1_W(y) \wedge EQ^2_W(y)$.

Case (3.3) does not hold:
Then one can choose $y = \ceilfrac{s_2+k_2+\varepsilon}{c_2}$ which refutes (1.3) and (2.3):

$y = \ceilfrac{s_2+k_2+\varepsilon}{c_2}$ in (2.3) immediately gives a contradiction:

\begin{align*} 
& - k_2 \leq -c_{2} \ceilfrac{s_2(\vec x)+k_2+\varepsilon}{c_2} +s_{2}(\vec x) \\
& \Leftrightarrow \ceilfrac{s_2(\vec x)+k_2+\varepsilon}{c_2} \leq \frac{s_2(\vec x) + k_2}{c_2} 
\end{align*}

For showing (1.3) we make use of the negation of (3.3):

\begin{align*} 
\text{negation of (3.3):} & s_1(\vec x)/c_1 + s_2(\vec x)/c_2 < -k_3 \\
\text{unfold $k_3$:} & 
  \Leftrightarrow \frac{s_1(\vec x)}{c_1} + \frac{s_2(\vec x)}{c_2} < -\frac{k_1}{c_1} - \frac{k_2}{c_2} - 1\\
  & \Leftrightarrow  \frac{s_2(\vec x) + k_2}{c_2} + 1 < \frac{- s_1(\vec x) - k_1}{c_1} \\
  & \Leftrightarrow \ceilfrac{s_2+k_2+\varepsilon}{c_2} < \frac{-s_1(x)-k_1}{c_1} \\
  & \Leftrightarrow c_1 \ceilfrac{s_2+k_2+\varepsilon}{c_2} + s_1(x) < -k_1
\end{align*}

The last line is exactly the negation of the instantiation of \text{inst $y = \ceilfrac{s_2+k_2+\varepsilon}{c_2}$ in (1.3)}.
\qed
\end{proof}

\begin{proof}
(real case)

Case (3.3) holds: 
This is perfectly analogous to the integer case. ($F_3$ holds, thus $EQ^1_W \wedge EQ^2_W$)

Case (3.3) does not hold:
Choose $y = \frac{s_2(x)+k_2+\varepsilon}{c_2}.$
Then $EQ^2_W(y)$ holds because (2.3) gives a contradiction:
\begin{align*}
  & -k_2 \leq -c_2\frac{s_2(x)+k_2+\varepsilon}{c_2}+s_2(x) \\
  &\Leftrightarrow s_2(x)+\varepsilon \leq s_2(x) \\
  &\Leftrightarrow 0<0
\end{align*}
Last we show that $EQ^1_W(y)$ holds because (1.3) and the negation of (3.3) (from the case split)
lead to a contradiction. To show this we need a second case split on the value
of $k_3$.
  
case $k_3 = \frac{k_1}{c_1} + \frac{k_2}{c_2}$:
\begin{align*}
  \text{\{negation of (3.3)\}} & \phantom{\Rightarrow} \frac{s_1(x)}{c_1} + \frac{s_2(x)}{c_2} 
  < - k_3 \\
  & \Rightarrow \frac{s_1(x)}{c_1} + \frac{s_2(x)}{c_2} 
  < - \frac{k_1}{c_1} - \frac{k_2}{c_2} \\
  & \Rightarrow \frac{s_2(x)+k_2}{c_2} < \frac{-s_1(x)-k_1}{c_1} & (*) \\
  \text{\{(1.3) with $y$ instantiated\}} & 
  \phantom{\Rightarrow} -k_1 \leq c_1 \frac{s_2(x)+k_2}{c_2}  + s_1(x) \\
  \text{\{$(*)$\}} & 
  \Rightarrow -k_1 < c_1 \frac{-s_1(x)-k_1}{c_1} + s_1(x) \\
  &\Rightarrow 0<0
\end{align*}

case $k_1=-\varepsilon, k_2\in \mathbb{Q}, k_3 = \frac{k_2}{c_2}$:
\begin{align*}
  \text{\{negation of (3.3)\}} & \phantom{\Rightarrow} \frac{s_1(x)}{c_1} + \frac{s_2(x)}{c_2} 
  < - k_3 \\
  & \Rightarrow \frac{s_1(x)}{c_1} + \frac{s_2(x)}{c_2} 
  < - \frac{k_2}{c_2} \\
  & \Rightarrow \frac{s_2(x)+k_2}{c_2} < \frac{-s_1(x)}{c_1} & (*) \\
  \text{\{(1.3) with $y$ instantiated, $k_1=-\varepsilon$\}} & 
  \phantom{\Rightarrow} \varepsilon \leq c_1 \frac{s_2(x)+k_2}{c_2}  + s_1(x) \\
  \text{\{$(*)$\}} & 
  \Rightarrow \varepsilon < c_1 \frac{-s_1(x)}{c_1} + s_1(x) \\
  &\Rightarrow \varepsilon<0
\end{align*}

case $k_1\in \mathbb{Q}, k_2=-\varepsilon,  k_3 = \frac{k_1}{c_1}$:
\begin{align*}
  \text{\{negation of (3.3)\}} & \phantom{\Rightarrow} \frac{s_1(x)}{c_1} + \frac{s_2(x)}{c_2} 
  < - k_3 \\
  & \Rightarrow \frac{s_1(x)}{c_1} + \frac{s_2(x)}{c_2} 
  < - \frac{k_1}{c_1} \\
  & \Rightarrow \frac{s_2(x)}{c_2} < \frac{-s_1(x)-k_1}{c_1} & (*) \\
  \text{\{(1.3) with $y$ instantiated, $k_1=-\varepsilon$\}} & 
  \phantom{\Rightarrow} -k_1 \leq c_1 \frac{s_2(x)+k_2}{c_2}  + s_1(x) \\
  \text{\{$(*),k_2=\varepsilon$\}} & 
  \Rightarrow -k_1 < c_1 (\frac{-s_1(x)-k_1}{c_1}-\varepsilon) + s_1(x) \\
  &\Rightarrow 0<-\varepsilon
\end{align*}

case $k_1=k_2=k_3=-\varepsilon$:
\begin{align*}
  \text{\{negation of (3.3)\}} & \phantom{\Rightarrow} \frac{s_1(x)}{c_1} + \frac{s_2(x)}{c_2} 
  < - k_3 \\
  & \Rightarrow \frac{s_1(x)}{c_1} + \frac{s_2(x)}{c_2} < \varepsilon \\
  & \Rightarrow \frac{s_2(x)}{c_2} < \frac{-s_1(x)}{c_1} + \varepsilon & (*) \\
  \text{\{(1.3) with $y$ instantiated, $k_1=k_2=-\varepsilon$\}} & 
  \phantom{\Rightarrow} \varepsilon \leq c_1 (\frac{s_2(x)}{c_2}+\varepsilon)  + s_1(x) \\
  \text{\{$(*)$\}} & 
  \Rightarrow \varepsilon < c_1 (\frac{-s_1(x)}{c_1} + \varepsilon - \varepsilon) + s_1(x) \\
  &\Rightarrow \varepsilon<0
\end{align*}
\phantom{phantom}
\qed
\end{proof}


\begin{theorem}
$I_1[I_2[EQ(c_2 s_1 + c_1 s_2, k_3, F_3)]$ has the contradictory property of a partial interpolant
of $A$ and $B$ at clause $C_1 \vee C_2$. 
\end{theorem}
\begin{proof}
To be shown: 
\begin{align*}
B\wedge\neg C_{1}|B \wedge \neg C_{2}|B \wedge I_1[I_2[EQ_W^3]] & \models \bot%\label{eq:10}
\end{align*}

Assume $B\wedge\neg C_1|B\wedge\neg C_2| B$, then we need to show $I_1[I_2[EQ_W^3]] \models \bot$.

By induction premise $I_1[EQ_W^1(y)]$ and $I_2[EQ_W^2(y)]$ are partial interpolants of $A$ and $B$ and 
thus have the following properties (for any $y$):
\begin{eqnarray}
y - b > 0 \wedge I_1[EQ_W^1(y)] & \models \bot \label{itpB1} \\
y - b \leq 0 \wedge I_2[EQ_W^2(y)] & \models \bot \label{itpB2} 
\end{eqnarray}

Remember lemma \ref{lemma_la2}:
\begin{align*}
  & EQ_W^3 \rightarrow \exists x. (EQ_W^1(x) \wedge EQ_W^2(x)) \\
  \Rightarrow & EQ_W^3 \rightarrow \exists x. EQ_W^1(x) \wedge \exists x. EQ_W^2(x)
\end{align*}

This and the third part of the deep substitution lemma give us:
\begin{align*}
I_1[I_2[EQ_W^3]] & \rightarrow (I_1[\exists x. EQ_W^1(x)] \wedge I_2[\exists x. EQ_W^2(x)]) \vee
I_1[\bot] \vee I_2[\bot]
\end{align*}

We prove for the three disjuncts in the succedent separately that they must be wrong, starting with:
\begin{align*}
& I_1[\exists x. EQ_W^1(x)] \wedge I_2[\exists x. EQ_W^2(x)] \stackrel{!}{\rightarrow} \bot \\
\Leftrightarrow & (I_1[\exists x. EQ_W^1(x)] \rightarrow \bot) \vee 
(I_2[\exists x. EQ_W^2(x)] \rightarrow \bot)
\end{align*}

If $y - b > 0$ holds, equation \ref{itpB1} gives us $(I_1(EQ_W^1(y)) \rightarrow \bot)$ for any $y$,
i.e. for any $y$, $I_1(EQ_W^1(y))$ is false, that means there is no $y$ which fulfills ist. So
$(I_1[\exists x. EQ_W^1(x)] \rightarrow \bot)$ holds. %TODO: zu umstaendlich erklaert??

Otherwise we can use equation \ref{itpB2} for asserting the second conjunct.

Next we prove $I_1[\bot] \stackrel{!}{\rightarrow} \bot$:

\begin{align*}
\text{\{ex falso quod libet\}} & \phantom{\Rightarrow} \bot \rightarrow EQ_W^1(y) \\
\text{\{monotonicity\}} & \Rightarrow I_1[\bot] \rightarrow I_1[EQ_W^1(y)] \\
\text{\{inst y:= $b + 1$\}} & \Rightarrow I_1[\bot] \rightarrow I_1[EQ_W^1(b + 1)] \\
\text{ \{equation \ref{itpB1} and transitivity\}} & \Rightarrow I_1[\bot] \rightarrow \bot 
\end{align*}

$I_2[\bot] \stackrel{!}{\rightarrow} \bot$ can be proven completely analogously with $y:=b$ and 
using equation (\ref{itpB2}).

\end{proof}


%----------------------------------------
\section{Conclusion}
%----------------------------------------

\end{document}
