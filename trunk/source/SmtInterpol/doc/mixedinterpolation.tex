\documentclass[a4paper]{article}
\usepackage{amsmath}
\usepackage{amsthm}
\usepackage{mathpartir}

\newtheorem{theorem}{Theorem}
\theoremstyle{definition}
\newtheorem{definition}{Definition}
\newtheorem{example}{Example}
\newtheorem{lemma}{Lemma}

\newcommand\syms{\mathop{\mathit{syms}}\nolimits}
\newcommand\restrictB{\downarrow B}
\newcommand\withoutB{\setminus B}
\newcommand\defas{:\equiv}
\newcommand\mR{\mathcal{R}}
\newcommand\mZ{\mathcal{Z}}
\newcommand\meq{\mathop{\mathit{EQ}}\nolimits}
\newcommand\meqr{\meq_\mR}
\newcommand\meqz{\meq_\mZ}
\newcommand\meqrone{\meq_\mR^{(1)}}
\newcommand\meqrtwo{\meq_\mR^{(2)}}
\newcommand\meqrthree{\meq_\mR^{(3)}}
\newcommand\meqzone{\meq_\mZ^{(1)}}
\newcommand\meqztwo{\meq_\mZ^{(2)}}
\newcommand\meqzthree{\meq_\mZ^{(3)}}

\begin{document}
\section{Properties of Partial Interpolants}
Partial interpolants are in NNF.
Partial interpolant $I$ is monotonic such that if $\phi_1\rightarrow\phi_2$holds, $I[\phi_1]\rightarrow I[\phi_2]$ holds, too.

\section{Mixed Equalities}
\begin{definition}
  Given $A$ and $B$ and an equality $a=b$ such that $a\not\in\syms(B)$ and $b\not\in\syms(A)$.
  We define
  \begin{eqnarray*}
    (a=b)\withoutB & \defas & a=x\\
    (a=b)\restrictB & \defas & x=b\\
    (a\ne b)\withoutB & \defas & a=x\\
    (a\ne b)\restrictB & \defas & x\ne b
  \end{eqnarray*}
\end{definition}

The pivoting rule for mixed equalities is
\[
\inferrule*{a=b\vee C_1 : I_1[x=s] \and a\ne b\vee C_2 : I_2(x)}{C_1\vee C_2 : I_1[I_2(s)]}
\]

\begin{proof}
  We have to show:
  \begin{eqnarray}
    A\wedge\neg C_1\withoutB\wedge\neg C_2\withoutB&\models&I_1[I_2(s)]\label{mixedeqfollowsfromA}\\
    B\wedge\neg C_1\restrictB\wedge\neg C_2\restrictB\wedge I_1[I_2(s)]&\models&\bot\label{mixedeqinconsistentwithB}
  \end{eqnarray}
  
  \paragraph{Proving \ref{mixedeqfollowsfromA}:}
  We have premises
  \begin{eqnarray}
    A\wedge\neg C_1\withoutB\wedge a=x&\models&I_1[x=s]\label{mixedeqap1}\\
    A\wedge\neg C_2\withoutB\wedge a=x&\models&I_2(x)\label{mixedeqap2}
  \end{eqnarray}

  Given $A\wedge\neg C_1\withoutB\wedge\neg C_2\withoutB$ we know from \ref{mixedeqap2} that $a=x\rightarrow I_2(x)$ is a tautology.
  Hence, it holds for all $x$ and, especially, for $x=s$.
  We get $a=s\rightarrow I_2(s)$.

  From \ref{mixedeqap1}, we know that $a=x\rightarrow I_1[x=s]$ is a tautology and holds for all values of $x$, especially for $x=a$.
  Hence, $I_1[a=s]$ holds.
  By monotonicity, we get $I_1[I_2(s)]$.
  This proves \ref{mixedeqfollowsfromA}.

  \paragraph{Proving \ref{mixedeqinconsistentwithB}:}
  We have premises
  \begin{eqnarray}
    B\wedge\neg C_1\restrictB\wedge x=b\wedge I_1[x=s]&\models&\bot\label{mixedeqbp1}\\
    B\wedge\neg C_2\restrictB\wedge x\ne b\wedge I_2(x)&\models&\bot\label{mixedeqbp2}
  \end{eqnarray}
  
  Given $B\wedge\neg C_1\restrictB\wedge\neg C_2\restrictB$, we know from \ref{mixedeqbp2} that $x\ne b\wedge I_2(x)\rightarrow\bot$ is a tautology.
  Since it holds for all $x$, it especially holds for $x=s$ yielding $s\ne b\wedge I_2(s)\rightarrow \bot$ or, equivalently, $I_2(s)\rightarrow b=s$.
  Since \ref{mixedeqbp1} holds for $x=b$, we get $I_1[b=s]\rightarrow\bot$.
  Hence, $I_1[I_2(s)]\rightarrow\bot$ holds:
  By monotonicity, since $I_2(s)\rightarrow b=s$, we get $I_1[I_2(s)]\rightarrow I_1[b=s]$ and $I_1[b=s]\rightarrow\bot$.
\end{proof}

\section{Mixed Inequalities}
\subsection{Rational Inequalities}
\begin{definition}\label{def:mixedineqsplit}
  Given $A$ and $B$ and two terms $a$ and $b$ such that $a\not\in\syms(B)$ and $b\not\in\syms(A)$.
  We define
  \begin{eqnarray*}
    (a\le b)\withoutB & \defas & a\le x\\
    (a\le b)\restrictB & \defas & x\le b\\
    (a>b)\withoutB & \defas & a\ge x\\
    (a>b)\restrictB & \defas & x>b
  \end{eqnarray*}
\end{definition}

\begin{definition}
  Let $s$ be a rational valued term and $F$ be a formula.
  We define $\meqr(s,F)\defas s\le 0\wedge(s<0\vee F)$.
\end{definition}

The pivoting rule for mixed inequalities is
\[
\inferrule*{t\le 0\vee C_1 : I_1[\meqr(c_1y+s_1,F_1)] \and t>0\vee C_2 : I_2[\meqr(-c_2y+s_2,F_2(y))]}{C_1\vee C_2 : I_1[I_2[\meqr(c_2s_1+c_1s_2,F_1(\frac{s_2}{c_2})\wedge F_2(\frac{s_2}{c_2}))]]}
\]
where $c_1,c_2>0$ are constants and $s_1,s_2$ are terms such that $\syms(s_i)\subseteq\syms(A)\cap\syms(B)$ for $i=1,2$.

\begin{proof}
  We have to show
  \begin{eqnarray}
    A\wedge\neg C_1\withoutB\wedge\neg C_2\withoutB\models I_1[I_2[\meqr(c_2s_1+c_1s_2,F_1(\frac{s_2}{c_2})\wedge F_2(\frac{s_2}{c_2}))]]&&\label{mixedineqrfollowsfromA}\\
    B\wedge\neg C_1\restrictB\wedge\neg C_2\restrictB\wedge I_1[I_2[\meqr(c_2s_1+c_1s_2,F_1(\frac{s_2}{c_2})\wedge F_2(\frac{s_2}{c_2}))]]\models\bot&&\label{mixedineqrinconsistentwithB}
  \end{eqnarray}

  \paragraph{Proving \ref{mixedineqrfollowsfromA}:}
  We define 
  \begin{eqnarray*}
    \meqrone&\defas&\meqr(c_1t_A+s_1,F_1(t_A))\\
    \meqrtwo&\defas&\meqr(-c_2t_A+s_2,F_2(t_A))\\
    \meqrthree&\defas&\meqr(c_2s_1+c_1s_2,F_1(\frac{s_2}{c_2})\wedge F_2(\frac{s_2}{c_2}))
    \end{eqnarray*}
  to improve readability.

  We have premises
  \begin{eqnarray}
    A\wedge\neg C_1\withoutB\wedge t_A-y\ge0&\models&I_1[\meqr(c_1y+s_1,F_1(y))]\label{mixedineqrap1}\\
    A\wedge\neg C_2\withoutB\wedge t_A-y\le0&\models&I_2[\meqr(-c_2y+s_2,F_2(y))]\label{mixedineqrap2}
  \end{eqnarray}
  
  Given $A\wedge C_1\withoutB\wedge C_2\withoutB$, we know that \ref{mixedineqrap1} and \ref{mixedineqrap2} have to hold for all possible values of $y$ and, especially, for $y=t_A$.
  Hence, we know that $\meqrone\wedge\meqrtwo$ hold.

  We divide the the proof into the following two steps:
  First, we show that $\meqrone\wedge\meqrtwo\rightarrow\meqrthree$ holds.
  Next, assuming this, we show that $I_1[\meqrone]\wedge I_2[\meqrtwo]\rightarrow I_1[I_2[\meqrthree]]$ holds.

  Given $\meqrone\wedge\meqrtwo$, we get
  \begin{eqnarray*}
    &&\meqrone\wedge\meqrtwo\equiv\\
    &&c_1t_A+s_1\le0\wedge(c_1t_A+s_1<0\vee F_1(t_A))\\
    &\wedge&-c_2t_A+s_2\le0\wedge(-c_2t_A+s_2<0\vee F_2(t_A))\equiv\\
    &&\frac{s_2}{c_2}\le t_A\wedge t_A\le-\frac{s_1}{c_1}\\
    &\wedge&(t_A<-\frac{s_1}{c_1}\vee F_1(t_A))\wedge(\frac{s_2}{c_2}<t_A\vee F_2(t_A))\\
  \end{eqnarray*}

  We define 
  \begin{eqnarray*}
    \phi&\defas&\frac{s_2}{c_2}\le t_A\wedge t_A\le-\frac{s_1}{c_1}\wedge(t_A<-\frac{s_1}{c_1}\vee F_1(t_A))\wedge(\frac{s_2}{c_2}<t_A\vee F_2(t_A))\\
    \psi&\defas&\frac{s_2}{c_2}\le-\frac{s_1}{c_1}\wedge(\frac{s_2}{c_2}<-\frac{s_1}{c_1}\vee(F_1(\frac{s_2}{c_2})\wedge F_2(\frac{s_2}{c_2})))
    \end{eqnarray*}
  and split cases on the (dis-)equality between $\frac{s_2}{c_2}$ and $-\frac{s_1}{c_1}$:\\
  \emph{Case 1:} $\frac{s_2}{c_2}\ne -\frac{s_1}{c_1}$:

  Given $\frac{s_2}{c_2}\le t_A\wedge t_A\le-\frac{s_1}{c_1}$, we get $\frac{s_2}{c_2}<-\frac{s_1}{c_1}$.
  Hence, if $\phi$ holds in this case, $\psi$ holds, too.

  \noindent\emph{Case 2:} $\frac{s_2}{c_2}=-\frac{s_1}{c_1}$:

  In this case, $t_A=\frac{s_2}{c_2}$ holds and, given $\phi$ we immediately get $\psi$.

  We get $\phi\rightarrow\psi$ holds and, thus, $\meqrone\wedge\meqrtwo\rightarrow\meqrthree$.
  
  The next part of the proof is almost trivial:
  \begin{eqnarray*}
    &&\meqrone\wedge\meqrtwo\rightarrow\meqrthree\\
    &\equiv&\meqrone\rightarrow(\meqrtwo\rightarrow\meqrthree)\\
    &\rightarrow&\meqrone\rightarrow(I_2[\meqrtwo]\rightarrow I_2[\meqrthree])\\
    &\equiv&\meqrone\wedge I_2[\meqrtwo]\rightarrow I_2[\meqrthree]\\
    &\equiv&I_2[\meqrtwo]\rightarrow(\meqrone\rightarrow I_2[\meqrthree]\\
    &\rightarrow&I_2[\meqrtwo]\rightarrow(I_1[\meqrone]\rightarrow I_1[I_2[\meqrthree]]\\
  \end{eqnarray*}

  Since \ref{mixedineqrap1} and \ref{mixedineqrap2} holds for every $y$, it especially holds for $y=t_A$.
  Hence, it get $A\wedge\neg C_1\withoutB\wedge\neg C_2\withoutB\models\phi$ and conclude that 
  \[
  A\wedge\neg C_1\withoutB\wedge\neg C_2\withoutB\models I_1[I_2[\meqr(c_1s_2+c_2s_1,F_1(\frac{s_2}{c_2})\wedge F_2(\frac{s_2}{c_2}))]]
  \]
  holds.

  \paragraph{Proving \ref{mixedineqrinconsistentwithB}:}
  We have premises
  \begin{eqnarray}
    B\wedge\neg C_1\restrictB\wedge y+t_B>0\wedge I_1[\meqr(c_1y+s_1,F_1(y))]&\models&\bot\label{mixedineqrbp1}\\
    B\wedge\neg C_2\restrictB\wedge y+t_B\le0\wedge I_2[\meqr(-c_2y+s_2,F_2(y))]&\models&\bot\label{mixedineqrbp2}
  \end{eqnarray}

  Applying resolution on \ref{mixedineqrbp1} and \ref{mixedineqrbp2} with pivot $y+t_B\le0$ yields
  \begin{equation}
    B\wedge\neg C_1\restrictB\wedge\neg C_2\restrictB\wedge I_1[\meqr(c_1y+s_1,F_1(y))]\wedge I_2[\meqr(-c_2y+s_2,F_2(y))]\models\bot\label{mixedineqrbintermediate1}
  \end{equation}

  We continue as follows:
  First, we show that $\meqr(c_1s_2+c_2s_1,F_1(\frac{s_2}{c_2})\wedge F_2(\frac{s_2}{c_2}))\rightarrow\exists y.\,\meqr(c_1y+s_1,F_1(y))\wedge\meqr(-c_2y+s_2,F_2(y))$ holds.
  Next, we show, using the implication previously shown, that, given $B\wedge\neg C_1\restrictB\wedge\neg C_2\restrictB$, we get \ref{mixedineqrinconsistentwithB}.

  Given $\meqr(c_1s_2+c_2s_1,F_1(\frac{s_2}{c_2})\wedge F_2(\frac{s_2}{c_2}))$, we choose $y=(\frac{s_2}{c_2}-\frac{s_1}{c_1})/2$.
  From $\meqr(c_1s_2+c_2s_1,F_1(\frac{s_2}{c_2})\wedge F_2(\frac{s_2}{c_2}))$ we know that $\frac{s_2}{c_2}\le-\frac{s_1}{c_1}$ holds and that either $\frac{s_2}{c_2}<-\frac{s_1}{c_1}$ or $F_1(\frac{s_2}{c_2})\wedge F_2(\frac{s_2}{c_2}))$ holds.
  We split cases on the (dis-)equality between $-\frac{s_1}{c_1}$ and $\frac{s_2}{c_2}$:\\
  \emph{Case 1:} $\frac{s_2}{c_2}=-\frac{s_1}{c_1}$:

  In this case, $y=\frac{s_2}{c_2}=-\frac{s_1}{c_1}$ holds and we obviously get $\meqr(c_1y+s_1,F_1(y))\wedge\meqr(-c_2y+s_2,F_2(y))$.

  \noindent\emph{Case 2:} $\frac{s_2}{c_2}\ne-\frac{s_1}{c_1}$:
  
  We get $\frac{s_2}{c_2}<y<-\frac{s_1}{c_1}$.
  Again, this gives $\meqr(c_1y+s_1,F_1(y))\wedge\meqr(-c_2y+s_2,F_2(y))$.

  Hence, $\meqr(c_1s_2+c_2s_1,F_1(\frac{s_2}{c_2})\wedge F_2(\frac{s_2}{c_2}))\rightarrow\exists y.\,\meqr(c_1y+s_1,F_1(y))\wedge\meqr(-c_2y+s_2,F_2(y))$ holds.
  In the following we suppress the existential quantifier since $y$ is then implicitly existentially quantified.

  Given $\meqr(c_1s_2+c_2s_1,F_1(\frac{s_2}{c_2})\wedge F_2(\frac{s_2}{c_2}))\rightarrow\meqr(c_1y+s_1,F_1(y))\wedge\meqr(-c_2y+s_2,F_2(y))$ and monotonicity of $I_1[\cdot]$ and $I_2[\cdot]$, we get
  \begin{eqnarray*}
    I_1[I_2[\meqr(c_1s_2+c_2s_1,F_1(\frac{s_2}{c_2})\wedge F_2(\frac{s_2}{c_2}))]]&\rightarrow&\\
    I_1[I_2[\meqr(c_1y+s_1,F_1(y))\wedge\meqr(-c_2y+s_2,F_2(y))]]&\rightarrow&\\
    I_1[\meqr(c_1y+s_1,F_1(y))\wedge I_2[\meqr(-c_2y+s_2,F_2(y))]]&\rightarrow&\\
    I_1[\meqr(c_1y+s_1,F_1(y))]\wedge I_2[\meqr(-c_2y+s_2,F_2(y))]&&
  \end{eqnarray*}

  From \ref{mixedineqrbintermediate1}, we get
  \[
  B\wedge\neg C_1\restrictB\wedge C_2\restrictB\wedge I_1[I_2[\meqr(c_1s_2+c_2s_1,F_1(\frac{s_2}{c_2})\wedge F_2(\frac{s_2}{c_2}))]]\models\bot
  \]

  Since we have shown \ref{mixedineqrfollowsfromA} and \ref{mixedineqrinconsistentwithB}, $I_1[I_2[\meqr(c_1s_2+c_2s_1,F_1(\frac{s_2}{c_2})\wedge F_2(\frac{s_2}{c_2}))]]$ is a valid partial interpolant for $C_1\vee C_2$.
\end{proof}

\subsection{Integer Inequalities}
We split mixed inequalities as presented in definition \ref{def:mixedineqsplit} for rational inequalities.
\begin{definition}
  Let $s$ be an integer valued term, $k$ be a non-negative integer, and $F$ be a formula. We define $\meqz(s,k,F)\defas s\le0\wedge(s<-k\vee F)$.
\end{definition}

The privoting rule for mixed integer inequalities is
\[
\inferrule*{t\le0\vee C_1 : I_1[\meqz(c_1y+s_1,k_1,F_1(y))] \and t > 0\vee C_2 : I_2[\meqz(-c_2y+s_2,k_2,F_2(y))]}{C_1\vee C_2 : I_1[I_2[\meqz(c_2s_1+c_1s_2,c_2k_1+c_1k_2+c_1c_2,F)]]}
\]
where $c_1,c_2>0$ and $k_1,k_2\ge0$ are integer constants and $s_1,s_2$ are terms such that $\syms(s_i)\subseteq\syms(A)\cap\syms(B)$ for $i=1,2$ and
\[
F\defas\exists x.\,\frac{s_2}{c_2}\le x\le-\frac{s_1}{c_1}\wedge(c_1x+s_1<-k_1\vee F_1(x))\wedge(-c_2x+s_2<-k_2\vee F_2(x))
\]
\begin{proof}
  We have to show
  \begin{eqnarray}
    A\wedge\neg C_1\withoutB\wedge\neg C_2\withoutB\models I_1[I_2[\meqz(c_2s_1+c_1s_2,c_1k_2+c_2k_1+c_1c_2,F)]]&&\label{mixedineqifollowsfromA}\\
    B\wedge\neg C_1\restrictB\wedge\neg C_2\restrictB\wedge I_1[I_2[\meqz(c_2s_1+c_1s_2,c_1k_2+c_2k_1+c_1c_2,F)]]\models\bot&&\label{mixedineqiinconsistentwithB}
  \end{eqnarray}

  \paragraph{Proving \ref{mixedineqifollowsfromA}:}
  We define
  \begin{eqnarray*}
    \meqzone(x)&\defas&\meqz(c_1x+s_1,k_1,F_1(x))\\
    \meqztwo(x)&\defas&\meqz(-c_2x+s_2,k_2,F_2(x))\\
    \meqzthree&\defas&\meqz(c_1s_2+c_2s_1,c_1k_2+c_2k_1+c_1c_2,F)
  \end{eqnarray*}

  We have premises
  \begin{eqnarray}
    A\wedge\neg C_1\withoutB\wedge t_A-y\ge0&\models&I_1[\meqzone(y)]\label{mixedineqzap1}\\
    A\wedge\neg C_2\withoutB\wedge t_A-y\le0&\models&I_2[\meqztwo(y)]\label{mixedineqzap2}
  \end{eqnarray}

  Given $A\wedge\neg C_1\withoutB\wedge\neg C_2\withoutB$, both premises \ref{mixedineqzap1} and \ref{mixedineqzap2} hold for every value of $y$, especially for $y=t_A$.
  Hence, $I_1[\meqzone(t_A)]\wedge I_2[\meqztwo(t_A)]$ holds.

  We show that $\meqzone(t_A)\wedge\meqztwo(t_A)\rightarrow\meqzthree$ holds and conclude \ref{mixedineqifollowsfromA} as we did for rational inequalities:
  \begin{eqnarray*}
    &&\meqzone(t_A)\wedge\meqztwo(t_A)\\
    &\rightarrow&c_1t_A+s_1\le0\wedge(c_1t_A+s_1<-k_1\vee F_1(t_A))\wedge-c_2t_A+s_2\le0\wedge(-c_2t_A+s_2<-k_2\vee F_2(t_A))\\
    &\rightarrow&c_1s_2+c_2s_1\le0\wedge F\\
    &\rightarrow&\meqzthree
  \end{eqnarray*}

  The pre-last implication holds since $F$ is the existential form of the antecedent and we can combine the unit clauses of the antecedent into the unit clause in this formula.

  Knowing that $\meqzone(t_A)\wedge\meqztwo(t_A)\rightarrow\meqzthree$ holds and that $I_1[\cdot]$ and $I_2[\cdot]$ are monotonic, we get
  \begin{eqnarray*}
    &&\meqzone(t_A)\wedge\meqztwo(t_A)\rightarrow\meqzthree\\
    &\equiv&\meqzone(t_A)\rightarrow(\meqztwo(t_A)\rightarrow\meqzthree)\\
    &\Rightarrow&\meqzone(t_A)\rightarrow(I_2[\meqztwo(t_A)]\rightarrow I_2[\meqzthree])\\
    &\equiv&\meqzone(t_A)\wedge I_2[\meqztwo(t_A)]\rightarrow I_2[\meqzthree]\\
    &\equiv&I_2[\meqztwo(t_A)]\rightarrow(\meqzone(t_A)\rightarrow I_2[\meqzthree])\\
    &\Rightarrow&I_2[\meqztwo(t_A)]\rightarrow(I_1[\meqzone(t_A)]\rightarrow I_1[I_2[\meqzthree]])\\
    &\equiv&I_2[\meqztwo(t_A)]\wedge I_1[\meqzone(t_A)]\rightarrow I_1[I_2[\meqzthree]]
  \end{eqnarray*}

  Since $A\wedge C_1\withoutB\wedge C_2\withoutB$ implies $I_1[\meqzone(t_A)]\wedge I_2[\meqztwo(t_A)]$, we get \ref{mixedineqifollowsfromA}.
  \paragraph{Proving \ref{mixedineqiinconsistentwithB}:}
\end{proof}
\end{document}
