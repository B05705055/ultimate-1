\documentclass[a4paper]{article}
\usepackage{amsmath}
\usepackage{amssymb}

\title{Interpolation for linear arithmetic}
\author{Jochen Hoenicke}

\begin{document}
\maketitle

\let\eps\varepsilon
\def\rational{\mathbb{Q}}
\def\nat{\mathbb{N}}

\def\basic{basic}      % better terms?
\def\rrestrictA{||{A}} % better notation
\def\restrictA{\setminus B}     % better notation
\def\restrictB{|B}     % better notation
\def\negc{\overline}
\def\lgeq{\lesseqgtr}
\def\insig{\prec}
\def\Var{\mathsf{Var}}
\def\implies{\Rightarrow}
\def\false{\mathsf{false}}
\def\round#1{[#1]}
\def\frac#1{\underline{#1}}

\section{Interpolation for Linear Arithmetic}

The basic algorithm used for interpolation is the McMillan/Pudl\'ak
algorithm in our DPLL-engine.  This algorithm needs an interpolant for
every conflict or unit clause generated by the theory.  In addition we
need to take care of mixed literals.  For these we need intermediate
interpolants that are resolved when a resolution step on the mixed
literals is performed.

\subsection{Mixed Literals}

Mixed literals are literals containing some variables local to A and some
variables local to B.  They are thus neither in the signature of the A formula
nor in the signature of the B formula.  This arises some problems for
the McMillan/Pudl\'ak algorithm used in our DPLL-engine.

Like all literals in our linear arithmetic setting, a mixed literal is of the
form $mix \sim c$, where $mix = \sum c_i x_i$, $x_i\in\Var$,
$c_i\in\rational$, $c \in \rational_{\eps}$ and $\sim\in\{\leq,\geq\}$.  In
our setting we introduce a fresh variable $x_{mix}$ for every mixed \basic
variable $mix$.  This variable is a placeholder for the A-local part 
$mix\rrestrictA := \sum_{x_i\in A \setminus B} c_i x_i$.  It is
regarded as shared symbol and may thus occur in the interpolant.  However, we
require that every appearance must be linked to a lower or upper bound
(depending on its sign) of $mix$ in the current conflict set.

The interpolant $I_C$ of a clause $C = \negc{\ell_1}\lor\dots\lor\negc{\ell_n}$ 
with conflict set $\ell_1 \land \dots\land \ell_n$ must fulfill:
\begin{eqnarray*}
  \ell_1\restrictA \land \dots \land \ell_n\restrictA &\implies I_C \\
  \ell_1\restrictB \land \dots \land \ell_n\restrictB &\land I_C \implies \bot
\end{eqnarray*}
where the $\ell\restrictA$ is
\[
\ell\restrictA := \begin{cases}
0 \lgeq 0 & \mbox{iff $\ell \insig B$}\\
\ell     & \mbox{iff $\ell \insig A, \ell\not\insig B$.}\\
mix\rrestrictA - x_{mix} \leq 0 & \mbox{iff $\ell = mix \leq c$, $mix$ is mixed} \\
mix\rrestrictA - x_{mix} \geq 0 & \mbox{iff $\ell = mix \geq c$, $mix$ is
  mixed}
\end{cases}
\]
and $\ell\restrictB = \ell - \ell\restrictA$.

When a resolution step on a mixed literal is performed, the algorithm needs to get rid of all mixed inequalities.  We compute the new interpolant as follows.
\begin{itemize}
\item We start with the first interpolant.
\item For every mixed inequality in the first interpolant we substitute the
  second interpolant.
\item For every mixed inequality in the inserted second interpolant, we combine it wiht the inequality of the first interpolant.
\end{itemize}

In the last step the inequalities of the two interpolants are
multiplied, so the mixed variable has the same absolute factor and
added.  Since the sign of the factor of the mixed variable differs,
the mixed variable is thus eliminated from the inequality.
Furthermore there may be a equality constraint that contains the mixed
variable.  In that case the mixed variable is set to one of the
constraints. TODO better explanation.


\subsection{Interpolation rules for conflict clauses}

The theory of linear arithmetic may generate two different conflict sets:
\begin{itemize}
\item Bound Propagation:
  \[ \sum_j c_{1,j} x_{j} \leq d_1 \land \dots \land
     \sum_j c_{n,j} x_{j} \leq d_n \]
  where there is a set of coefficients $c_i > 0$, with 
  $\sum_i c_ic_{i,j} = 0$ for
  all $j$, $\sum_i{c_i\cdot d_i} < 0$.  The interpolant
  is
  \[ I := \sum_i c_i (\sum_j c_{i,j} x_j \leq d_i)\restrictA \]
  
\item Bound refinement Propagation: 
  \[ y \leq c_1 \land y \geq c_2 \quad\mbox{where $c_2 < c_1$} \]
  this is a special case, of bound propagation.
\end{itemize}


\subsection{Gomory's cuts and cut interpolation}

A Gomory's cut comes from a row of the tableaux:
\[ x = \sum_i c_i y_i = \sum_i \sum_j c_i{c_{ij}}x_j \]
where $\beta(x) \in\rational\setminus\nat$ and for all $y_i$ with
$c_i\notin\nat$, $\beta(y_i)$ is either the upper or lower bound. 
We first bring everything integer to the left-hand-side:

\[ x - \sum \round{c_i} y_i - \round{\sum_i \frac{c_i}\beta(y_i)}= 
   \sum_i \frac{c_i} (y_i - \beta(y_i))  + \frac{\sum_i \frac{c_i}\beta(y_i)}\]
where $\round{c}$ is either $\lceil
c\rceil$ or $\lfloor c\rfloor$ depending on some simple heuristics and
$\frac{c} = \round{c} - c$.
We define $f := \frac{\sum_i \frac{c_i}\beta(y_i)} = \frac{\beta(x)}$, $0 < f
< 1$ and denote the left-hand-side by $iv$.

We further split $\sum_i \frac{c_i} (y_i - \beta(y_i))$ into three parts: $eq$
if $\beta(y_i)$ is the lower and upper bound of $y_i$, $inc$ if $\beta(y_i)$
is lower bound and $\frac{c_i}$ is positive or $\beta(y_i)$ is upper bound and
$\frac{c_i}$ is negative, and $dec$ otherwise.  Then the equation becomes
\begin{equation}\label{iveqincdeceq}
   iv = inc + dec + eq + f
\end{equation}
where $iv$ is a sum integer variables and constants, the current bounds
require $inc \geq 0$, $dec \leq 0$, $eq=0$, and the value $f$ is a constant
with $0< f < 1$.  Define $f' = f-1$, then the Gomory's cut can be written as:
\[ f\cdot inc + f' \cdot dec + f\cdot f' \geq 0 \]

The cut holds for the following reason:  Assume the cut would not hold.  Then:
\begin{align*}
 0 &> f\cdot inc + f'\cdot dec + f\cdot f' + dec + f\cdot eq
   =    f\cdot(iv - 1) \\
 0 &> f\cdot inc + f'\cdot dec + f'\cdot f - inc + f'\cdot eq
   =   f'\cdot(iv) \\
\end{align*}
The first equation implies $0 > (iv-1)$ hence $iv < 1$, the second equation
implies $0 < iv$ (since $f'$ is negative).  This contradicts that $iv$ must
have an integer value.

For the interpolant of the cut we introduce a new variable $y_{cut}$ which
stands for $iv\rrestrictA$.  If we restrict equation~(\ref{iveqincdeceq}) to
the A local variables, we get:
\begin{eqnarray*}
  f  (iv\rrestrictA) &= (\negc{cut}\rrestrictA) + (dec\rrestrictA) + f(eq\rrestrictA)\\
  f' (iv\rrestrictA) &= (\negc{cut}\rrestrictA) - (inc\rrestrictA) + f'(eq\rrestrictA)
\end{eqnarray*}

Hence the following interpolant,
which follows from the negated cut formula and the bound formulas $inc\geq0$,
$dec\leq 0$, $eq = 0$,
contains no A local variable:
\begin{align*}
&\exists y_{cut}. \\
&\quad  \negc{cut}\restrictA + (dec \leq 0) \restrictA + f(eq = 0)\restrictA
  + f(y_{cut} - iv\rrestrictA = 0)\\
&\land \\
&\quad  \negc{cut}\restrictA - (inc \geq 0) \restrictA + f'(eq = 0)\restrictA
  + f'(y_{cut} - iv\rrestrictA = 0).
\end{align*}

\end{document}
